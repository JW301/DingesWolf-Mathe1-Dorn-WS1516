% !TEX TS-program = pdflatex
% !TEX encoding = UTF-8 Unicode

\documentclass[a4paper, 12pt, twoside] {article}

\usepackage[utf8]{inputenc} % set input encoding (not needed with XeLaTeX)
\usepackage[ngerman]{babel}

\usepackage{amssymb} % math stuff
\usepackage{amsmath} % math stuff
\usepackage{multicol}

\usepackage{geometry} 
\usepackage{graphicx}

% Overwrite symbols in author footnotes
\makeatletter
\let\@fnsymbol\@arabic
\makeatother

\usepackage[parfill]{parskip} % Activate to begin paragraphs with an empty line rather than an indent

%%% PACKAGES
\usepackage{booktabs} % for much better looking tables
\usepackage{array} % for better arrays (eg matrices) in maths
\usepackage{paralist} % very flexible & customisable lists (eg. enumerate/itemize, etc.)
\usepackage{verbatim} % adds environment for commenting out blocks of text & for better verbatim
\usepackage{subfig} % make it possible to include more than one captioned figure/table in a single float
% These packages are all incorporated in the memoir class to one degree or another...

%%% HEADERS & FOOTERS
\usepackage{fancyhdr} % This should be set AFTER setting up the page geometry
\pagestyle{fancy} % options: empty , plain , fancy
\renewcommand{\headrulewidth}{0pt} % customise the layout...
\lhead{}\chead{}\rhead{}
\lfoot{}\cfoot{\thepage}\rfoot{}

%%% SECTION TITLE APPEARANCE
\usepackage{sectsty}
\allsectionsfont{\sffamily\mdseries\upshape} % (See the fntguide.pdf for font help)
% (This matches ConTeXt defaults)

%%% ToC (table of contents) APPEARANCE
\usepackage[nottoc,notlof,notlot]{tocbibind} % Put the bibliography in the ToC
\usepackage[titles,subfigure]{tocloft} % Alter the style of the Table of Contents
\renewcommand{\cftsecfont}{\rmfamily\mdseries\upshape}
\renewcommand{\cftsecpagefont}{\rmfamily\mdseries\upshape} % No bold!

\usepackage{wasysym}

\usepackage{tikz}

\usepackage{venndiagram}

\usepackage{mathtools}

\usepackage{commath}

\usepackage{hyperref} % This should be the last package loaded.

\hypersetup{linktoc=all,  
hidelinks}


\DeclarePairedDelimiter\ceil{\lceil}{\rceil}
\DeclarePairedDelimiter\floor{\lfloor}{\rfloor}

\newcommand{\attention}{{\fontencoding{U}\fontfamily{futs}\selectfont\char 66\relax}\space}

%%% END Article customizations

%%% CONTENT starts here

\title{Mathematik I WS 15/16}
\author{Thomas Dinges\thanks{thomas.dinges@student.uni-tuebingen.de} \and Jonas Wolf \thanks{mail@jonaswolf.de}}

\begin{document}
\maketitle

\vfill
\thanks{Inoffizielles Skript für die Vorlesung Mathematik I im WS 15/16, bei Britta Dorn. Alle Angaben ohne Gewähr. Fehler können gerne via E-Mail gemeldet werden.}

\newpage

\tableofcontents

\newpage

%%%%%%%%%%%%%%%%%%%%%%%%%%%%%%%
% Kapitel 1: Logik
%%%%%%%%%%%%%%%%%%%%%%%%%%%%%%%
\section{Logik}

% =============
% 12 Oktober 2015
% =============

\subsection*{Aussagenlogik}
Eine \textbf{logische Aussage} ist ein Satz, der entweder wahr oder falsch (also nie beides zugleich) ist. 
Wahre Aussagen haben den Wahrheitswert 1 (auch wahr, w, true, t), falsche den Wert 0 (auch falsch, f, false).

Notation: Aussagenvariablen $A, B, C, ... A_1, A_2$.

Beispiele:
\begin{itemize}
\item 2 ist eine gerade Zahl (1)
\item Heute ist Montag (1)
\item 2 ist eine Primzahl (1)
\item 12 ist eine Primzahl (0)
\item Es gibt unendlich viele Primzahlen (1)
\item Es gibt unendlich viele Primzahlzwillinge (Aussage, aber unbekannt, ob 1 oder 0)
\item 7 (keine Aussage)
\item Ist 173 eine Primzahl? (keine Aussage)
\end{itemize}

% =============
% 14 Oktober 2015
% =============

Aus einfachen Aussagen kann man durch logische Verknüpfungen (\textbf{Junktoren}, z.B. und, oder, ...) kompliziertere bilden. Diese werden Ausdrücke genannt (auch Aussagen sind Ausdrücke). 
Durch sogenannte \textbf{Wahrheitstafeln} gibt man an, wie der Wahrheitswert der zusammengesetzten Aussage durch die Werte der Teilaussagen bedingt ist. Im folgenden seien $A, B$ Aussagen. 

Die wichtigsten Junktoren:

\subsection{Negation}
Verneinung von A: $\neg A$ (auch $\bar{A})$, \textit{nicht A}, ist die Aussage, die genau dann wahr ist, wenn A falsch ist.

Wahrheitstafel: \qquad
\begin{tabular}{| c | c |}
\hline
A & $\neg A$ \\
\hline
1 & 0 \\
0 & 1 \\
\hline
\end{tabular}

Beispiele: 
\begin{itemize}
\item $A$: 6 ist durch 3 teilbar. (1)
\item $\neg A $: 6 ist nicht durch 3 teilbar. (0)
\item $B$: 4,5 ist eine gerade Zahl (0)
\item $\neg B$: 4,5 ist keine gerade Zahl. (1)
\end{itemize}

\subsection{Konjunktion}
Verknüpfung von A und B durch \textit{und}: $A \wedge B$ ist genau dann wahr, wenn A und B gleichzeitig wahr sind.

Wahrheitstafel: \qquad
\begin{tabular}{| c c | c |}
\hline
A & B & $A \wedge B$ \\
\hline
1 & 1 & 1 \\
1 & 0 & 0 \\
0 & 1 & 0 \\
0 & 0 & 0 \\
\hline
\end{tabular}

Beispiele:
\begin{itemize}
\item $\underbrace{\text{6 ist eine gerade Zahl}}_{A (1)}$ und $\underbrace{\text{durch 3 teilbar}}_{B (1)}$. (1)
\item $\underbrace{\text{9 ist eine gerade Zahl}}_{A (0)}$ und $\underbrace{\text{durch 3 teilbar}}_{B (1)}$. (0)
\end{itemize}

\subsection{Disjunktion}
\textit{oder}: $A \lor B$

Wahrheitstafel: \qquad
\begin{tabular}{| c c | c |}
\hline
A & B & $A \lor B$ \\
\hline
1 & 1 & 1 \\
1 & 0 & 1 \\
0 & 1 & 1 \\
0 & 0 & 0 \\
\hline
\end{tabular}

\attention Einschließendes oder, kein entweder...oder.

Beispiele:
\begin{itemize}
\item 6 ist gerade oder durch 3 teilbar. (1)
\item 9 ist gerade oder durch 3 teilbar. (1)
\item 7 ist gerade oder durch 3 teilbar. (0)
\end{itemize}

\subsection{XOR}
\textit{entweder oder}: A xor B, $A \oplus B$ (ausschließendes oder, exclusive or).

Wahrheitstafel: \qquad
\begin{tabular}{| c c | c |}
\hline
A & B & $A \oplus B$ \\
\hline
1 & 1 & 0 \\
1 & 0 & 1 \\
0 & 1 & 1 \\
0 & 0 & 0 \\
\hline
\end{tabular}

\subsection{Implikation}
\textit{wenn, dann}, $A \Rightarrow B$:
\begin{itemize}
\item wenn A gilt, dann auch B
\item A impliziert B
\item aus A folgt B
\item A ist \underline{hinreichend} für B,
\item B ist \underline{notwendig} für A
\end{itemize}

Wahrheitstafel: \qquad
\begin{tabular}{| c c | c |}
\hline
A & B & $A \Rightarrow B$ \\
\hline
1 & 1 & 1 \\
1 & 0 & 0 \\
0 & 1 & 1 \\
0 & 0 & 1 \\
\hline
\end{tabular}

\textbf{Merke: } \textit{ex falso quodlibet} : aus einer falschen Aussage kann man alles folgern!

(Die Implikation $A \Rightarrow B$ sagt nur, dass B wahr sein muss, \underline{falls} A wahr ist. Sie sagt nicht, dass B tatsächlich war ist.)

Beispiele:
\begin{itemize}
\item Wenn 1 = 0, bin ich der Papst. (1)
\end{itemize}

\subsection{Äquivalenz}
\textit{genau dann wenn}, $ A \Leftrightarrow B$ (dann und nur dann wenn, g.d.w, äquivalent, if and only if, iff)

Wahrheitstafel: \qquad
\begin{tabular}{| c c | c |}
\hline
A & B & $A \Leftrightarrow B$ \\
\hline
1 & 1 & 1 \\
1 & 0 & 0 \\
0 & 1 & 0 \\
0 & 0 & 1 \\
\hline
\end{tabular}

Beispiele:
\begin{itemize}
\item Heute ist Montag genau dann wenn morgen Dienstag ist. (1)
\item Eine natürliche Zahl $\underbrace{\text{ist durch 6 teilbar}}_{A}$ g. d. w. sie $\underbrace{\text{durch 3 teilbar ist}}_{B}$. (0) 
$A \Rightarrow B$ (1) 

$B \Rightarrow A$ (0)
\end{itemize}

% =============
% 19 Oktober 2015
% =============

\subsection*{Festlegung}
$\neg$ bindet stärker als alle anderen Junktoren: $(\neg A \wedge B)$ heißt $ (\neg A) \wedge B$

\subsection{Beispiel}
\subsubsection*{a)}
Wann ist der Ausdruck $(A \lor B) \wedge \neg (A \wedge B)$ wahr?

$\rightarrow$ Wahrheitstafel

\begin{tabular}{| c c | c | c | c | c |}
\hline
A & B & $(A \lor B)$ & $(A \wedge B)$ & $\neg (A \wedge B)$  & $(A \lor B) \wedge \neg (A \wedge B)$ \\
\hline
1 & 1 & 1 & 1 & 0 & 0 \\
1 & 0 & 1 & 0 & 1 & 1 \\
0 & 1 & 1 & 0 & 1 & 1 \\
0 & 0 & 0 & 0 & 1 & 0 \\
\hline
\end{tabular}

\attention Klammerung relevant

Welche Wahrheitswerte ergeben sich für
\begin{itemize}
\item $A \lor (B \wedge \neg A) \wedge B)$?
\item $A \lor B \wedge \neg A \wedge B$?
\end{itemize}

$(A \lor B) \wedge \neg (A \wedge B)$ und $(A \oplus B)$ haben dieselben Wahrheitstafeln.
Ausdrücke sehen unterschiedlich aus (Syntax), aber haben dieselbe Bedeutung (Semantik). Dies führt zu \textit{1.8 Definition}.

\subsubsection*{b)}
Wann ist $(A \wedge B) \Rightarrow \neg (C \lor A)$ falsch?

$\rightarrow$ Wahrheitstafel:
\underline{alle} möglichen Belegungen von $A, B, C$ mit $0 / 1$

\begin{tabular}{| c c c | c | c | c |}
\hline
A & B & C & $(A \wedge B)$ & $\neg(C \lor A)$ & $(A \wedge B) \Rightarrow \neg (C \lor A)$ \\
\hline
1 & 1 & 1 & 1 & 0 & 0 \\
1 & 1 & 0 & 1 & 0 & 0 \\
1 & 0 & 1 & 0 & 0 & 1 \\
1 & 0 & 0 & 0 & 0 & 1 \\
0 & 1 & 1 & 0 & 0 & 1 \\
0 & 1 & 0 & 0 & 1 & 1 \\
0 & 0 & 1 & 0 & 0 & 1 \\
0 & 0 & 0 & 0 & 1 & 1 \\
\hline
\end{tabular}

oder überlegen:

$(A \wedge B) \Rightarrow \neg  (C \lor A)$ ist nur 0, wenn

\qquad $(A \wedge B) = 1$, also $A = 1$ und $B = 1$

und

\qquad $\neg(C \lor A) = 0$ ist.

(Wissen: $A = 1$), also $\underline{C = 0}$ oder $ \underline{C = 1}$ möglich. 

\subsection{Definition}

Haben zwei Ausdrücke $\alpha$ und $\beta$ bei jeder Kombination von Wahrheitswerten ihrer Aussagevariablen den gleichen Wahrheitswert, so heißen sie \underline{logisch äquivalent}; man schreibt $\alpha \equiv \beta$. ('$\equiv$' ist kein Junktor, entspricht '$=$')

Es gilt: Falls $\alpha \equiv \beta$ gilt, hat der Ausdruck $\alpha \Leftrightarrow \beta$ immer den Wahrheitswert $1$.

\subsection{Satz}

Seien $A$, $B$, $C$ Aussagen.
Es gelten folgende logische Äquivalenzen:
\begin{description}
  \item[a) Doppelte Negation:]
  $A \equiv \neg(\neg A)$

  \item[b) Kommutativität von $\wedge$, $\lor$, $\oplus$, $\Leftrightarrow$:] \hfill
  \begin{itemize}
    \item $(A \wedge B) \equiv (B \wedge A)$
    \item $(A \lor B) \equiv (B \lor A)$
    \item $(A \oplus B) \equiv (B \oplus A)$
    \item $(A \Leftrightarrow B) \equiv (B \Leftrightarrow A)$

    \attention gilt nicht für '$\Rightarrow$' !! ($A \Rightarrow B \not\equiv B \Rightarrow A$)
  \end{itemize}  

  \item[c) Assoziativität von $\wedge$, $\lor$, $\oplus$, $\Leftrightarrow$:] \hfill
  \begin{itemize}
      \item $(A \wedge B) \wedge C \equiv A \wedge (B \wedge C)$
      \item $(A \lor B) \lor C \equiv A \lor (B \lor C)$
      \item $(A \oplus B) \oplus C \equiv A \oplus (B \oplus C)$
      \item $(A \Leftrightarrow B) \Leftrightarrow C \equiv A \Leftrightarrow (B \Leftrightarrow C)$
  \end{itemize}

  \item[d) Distributivität:] \hfill
  \begin{itemize}
  \item $A \wedge (B \lor C) \equiv (A \wedge B) \lor (A \wedge C)$
  \item $A \lor (B \wedge C) \equiv (A \lor B) \wedge (A \lor C)$
  \end{itemize}

  \item[e) Regeln von DeMorgan:] \hfill
  \begin{itemize}
  \item $\neg (A \wedge B) \equiv \neg A \lor \neg B$
  \item $\neg (A \lor B) \equiv \neg A \wedge \neg B$
  \end{itemize}

  \item[f)]
  $A \Rightarrow B \equiv \neg B \Rightarrow \neg A$

  \item[g)]
  $A \Rightarrow B \equiv \neg A \lor B$

  \item[h)]
  $A \Leftrightarrow B \equiv (A \Rightarrow B) \wedge (B \Rightarrow A)$
\end{description}
  (Alle Äquivalenzen gelten auch, wenn die Aussagevariablen durch Ausdrücke ersetzt werden.)

\underline{Beweis:} Jeweils mittels Wahrheitstafel (Übung!), zum Beispiel:

a) \qquad
\begin{tabular}{| c | c | c |}
\hline
A & $\neg A$ & $\neg (\neg A)$ \\
\hline
1 & 0 & 1 \\
0 & 1 & 0 \\
\hline
\end{tabular}

%% MISSING: arrows to show identity of columns 0 and 2

e) \qquad
\begin{tabular}{| c c | c | c | c | c | c |}
\hline
A & B & $(A \wedge B)$ & $\neg (A \wedge B)$ & $\neg A$ & $\neg B$ & $(\neg A \lor \neg B)$ \\
\hline
1 & 1 & 1 & 0 & 0 & 0 & 0 \\
1 & 0 & 0 & 1 & 0 & 1 & 1 \\
0 & 1 & 0 & 1 & 1 & 0 & 1 \\
0 & 0 & 0 & 1 & 1 & 1 & 1 \\
\hline
\end{tabular}

%% MISSING: arrows to show identity of columns 3 and 6

\subsection{Bemerkung}
(1.9 f): $(A \Rightarrow B) \equiv \underbrace{(\neg B \Rightarrow \neg A)}_{\mathrlap{\text{wird \underline{Kontraposition} genannt, wichtig für Beweis. Wird im Sprachgebrauch oft falsch verwendet.}}}$

\hfill

\textbf{Beispiel:} $\underset{A}{\text{Pit ist ein Dackel.}} \Rightarrow \underset{B}{\text{Pit ist ein Hund.}}$

äquivalent zu: $(\neg B) \Rightarrow (\neg A)$

\qquad Pit ist kein Hund. $\Rightarrow$ Pit ist kein Dackel.

aber nicht zu: $B \Rightarrow A$

\qquad Pit ist ein Hund. $\Rightarrow$ Pit ist ein Dackel.

und nicht zu: $\neg A \Rightarrow \neg B$

\qquad Pit ist kein Dackel. $\Rightarrow$ Pit ist kein Hund.

\textbf{Beispiel:} Sohn des Logikers / bellende Hunde ($\rightarrow$ Folien)

\subsection{Bemerkung (Logisches Umformen)}
Sei $\alpha$ ein Ausdruck. Ersetzen von Teilausdrücken von $\alpha$ durch logisch äquivalente Ausdrücke liefert einen zu $\alpha$ äquivalenten Ausdruck. So erhält man eventuell kürzere/einfachere Ausdrücke, zum Beispiel:

$\neg (A \Rightarrow B) \underset{\text{1.9 g})}{\equiv} \neg (\neg A \lor B) \underset{\text{1.9 e)}}{\equiv} \neg (\neg A) \wedge (\neg B) \underset{\text{1.9 a)}}{\equiv} A \wedge \neg B$

% =============
% 21 Oktober 2015
% =============

\subsection{Definition}
Ein Ausdruck heißt \underline{Tautologie}, wenn er für jede Belegung seiner Aussagevariablen, immer den Wert 1 annimmt. Hat er immer Wert 0, heißt er \underline{Kontradiktion}. 
Gibt es mindestens eine Belegung der Aussagevariablen, so dass der Ausdruck Wert 1 hat, heißt er \underline{erfüllbar}.

\subsection{Beispiel}
\begin{itemize}
\item[a)] $A \lor \neg A$ Tautologie \newline $A \wedge \neg A$ Kontradiktion

\item[b)] $\neg (A \Rightarrow B ) \Leftrightarrow A \wedge \neg B$ Tautologie (vergleiche Beispiel in 1.11). \newline
$(A \Rightarrow B) \Leftrightarrow (\neg A \lor B)$ Tautologie (vergleiche Beispiel in 1.9g).

\item[c)] $A \wedge \neg B$ ist erfüllbar (durch $A = 1, B = 0$).
\end{itemize}

\subsection*{Prädikatenlogik}
Eine \underline{Aussageform} ist ein sprachliches Gebilde, dass formal wie eine Aussage aussieht, aber eine oder mehrere Variablen enthält.

Beispiel:
$P(x): \underbrace{x}_{Variable} \underbrace{< 10}_{\mathrlap{\text{Prädikat (Eigenschaft)}}}$

$Q(x): x$ studiert Informatik
$R(y): y$ ist Primzahl und $y^2+2$ ist Primzahl.

Eine Aussageform$P(x)$ wird zur Aussage, wenn man die Variable durch ein konkretes Objekt ersetzt. Diest ist nur dann sinnvoll, wenn klar ist, welche Werte für x erlaubt sind, daher wird oft die zugelassene Wertemenge mit angegeben. (hier Vorgriff auf Kapitel \textit{Mengen})

Im Beispiel:

$P(3)$ ist wahr, $P(42)$ falsch.

$R(2)$ ist falsch, $R(3)$ ist wahr.

Oft ist die Frage interessant, ob es wenigstens ein $x$ gibt, für das $P(x)$ wahr ist, oder ob $P(x)$ sogar für alle zugelassenen $x$ wahr ist.

\subsection{Definition}
Sei $P(x)$ eine Aussageform.

a) Die Aussage \textit{Für alle x (aus einer bestimmten Menge M) gilt $P(x)$.} ist wahr genau dann wenn $P(x)$ für alle in Frage kommenden $x$ wahr ist.

Schreibweise: $\underbrace{\forall}_{\text{für alle, für jedes}} x \underbrace{\in M}_{\text{aus der Menge M}} \underbrace{:}_{\text{gilt}} \underbrace{P(x)}_{\text{Eigenschaft}}$

auch $\underbrace{\forall}_{x \in M} P(x)$.

Das Symbol $\forall$ heißt All- Quantor, die Aussage All- Aussage.

b) Die Aussage \textit{Es gibt (mindestens) ein x aus M, das die Eigenschaft P(x) besitzt.} ist wahr, g.d.w P(x) für mindestens eines der in Frage kommenden x wahr ist.

Schreibweise: $\underbrace{\exists}_{\text{es gibt, es existiert}} x \in M \underbrace{:}_{\text{so dass gilt}} P(x)$.

$\exists$ heißt Existenzquantor, die Aussage Existenzmenge.

\subsection{Beispiel / Bemerkung}
Übungsgruppe G:
$\underbrace{a}_{Anna} \underbrace{b}_{Bob} \underbrace{c}_{Clara}$

$B(x): x$ ist blond.
$W(x): x$ ist weiblich.

$B(a) = 1, W(b) = 0)$

\begin{enumerate}

\item Alle Studenten der Gruppe sind blond. (1)

$\forall x \in G$: x ist blond

$\forall x \in G$: B(x) (1)

Das bedeutet:
a blond $\wedge$ b blond $\wedge$ c blond \newline
$\underbrace{B(a)}_{1} \wedge \underbrace{B(b)}_{1} \wedge \underbrace{B(c)}_{1}$

$\forall$ ist also eine Verallgemeinerung der Konjunktion.

\item Alle Studenten der Gruppe sind weiblich. (0)

$\underbrace{W(a)}_{1} \wedge \underbrace{W(b)}_{0} \wedge \underbrace{W(c)}_{1} (0)$

\item Es gibt einen Studenten der Gruppe, der weiblich ist. (1)

$\exists x \in G$: W(x) (1)

bedeutet: $\underbrace{W(a)}_{1} \lor \underbrace{W(b)}_{0} \lor \underbrace{W(c)}_{1} = 1$

$\exists$ ist verallgemeinerte Disjunktion.

\item Aussage A: Alle Studenten der Gruppe sind weiblich. (0)

Verneinung von A? $\neg A$

\attention Nicht korrekt wäre: Alle Studenten der Gruppe sind männlich. (Wahrheitswert ist auch 0)

Korrekt: Nicht alle Studenten der Gruppe sind weiblich (1)
Es gibt (mindestens) einen Studenten der Gruppe, der nicht weiblich ist. (1)

\end{enumerate}

allgemeiner:

\subsection{Negation von All- und Existenzaussagen}

\begin{itemize}
\item[a)] $\neg (\forall x \in M: P(x)) \equiv \exists x \in M: \neg P(x)$
\item[b)] $\neg (\exists x \in M: P(x)) \equiv \forall x \in M : \neg P(x)$
\end{itemize}

(Verallgemeinerung der Regeln von DeMorgan)
(vergleiche Beispiel 1.15, 4):

$\neg (\forall x \in G: W(x))$

$\equiv \neg (W(a) \wedge W(b) \wedge W(c)$

$\underbrace{\equiv}_{\mathrlap{DeMorgan}} (\neg W(a)) \lor (\neg W(b)) \lor (\neg (W(c))$

$\equiv \exists x \in G: \neg W(x)$

% =============
% 26 Oktober 2015
% =============

\subsection*{Bemerkung}
Aussageformen können auch mehrere Variablen enthalten, Aussagen mit mehreren Quantoren sind möglich.

Zum Beispiel:

$\exists x \in X \quad \exists y \in Y: P(x,y)$ \\
$\exists x \in X \quad \forall y \in Y: P(x,y)$ \\
$\forall x \in X \quad \exists y \in Y: P(x,y)$ \\
$\forall x \in X \quad \forall y \in Y: P(x,y)$

Negation dann durch mehrfaches Anwenden von 1.16, zum Beispiel:

$\neg (\forall x \in X \quad \forall y \in Y \quad \exists z \in Z : P(x,y,z))$ \\
$\equiv \exists x \in X : \neg (\forall y \in Y \quad \exists z \in Z : P(x,y,z))$ \\
$\equiv \exists x \in X \quad \exists y \in Y : \neg (\exists z \in Z : P(x,y,z))$ \\
$\equiv \exists x \in X \quad \exists y \in Y \quad \forall z \in Z : \neg P(x,y,z))$

\textbf{Also: }\\
ändere $\exists$ in $\forall$, \\
\text{\qquad \quad} $\forall$ in $\exists$, \\
verneine Prädikat.

%%%%%%%%%%%%%%%%%%%%%%%%%%%%%%%
% Kapitel 2: Mengen
%%%%%%%%%%%%%%%%%%%%%%%%%%%%%%%
\section{Mengen}

\subsection{Definition (Georg Cantor, 1845-1918)}

Eine \underline{Menge} ist eine Zusammenfassung von bestimmten wohlunterscheidbaren Objekten (\underline{Elementen}) unserer Anschauung oder unseres Denkens zu einem Ganzen.

Im Folgenden seien $A$, $B$ Mengen.

\begin{description}
\item[a)] 
	$\quad x \in A : x \text{ ist Element der Menge } A$ \\
	$x \notin A: x \text{ ist nicht Element der Menge } A$ \\
	oder auch: \\
	$A \ni x : x \text{ ist Element der Menge } A$ \\
	$A \not \ni x: x \text{ ist nicht Element der Menge } A$
\item[b)]
	Eine Menge kann beschrieben werden durch:
	\begin{itemize}
		\item Aufzählung ihrer Elemente, zum Beispiel: \\
		$M_1 = \{a,b,c\} \qquad \text{(}=\{c,a,b\} \text{, d.h. Reihenfolge spielt keine Rolle)}$ \\
		\textbf{Achtung:} Keine Wiederholungen! \\
		$M_2 = \{\smiley,\frownie\}$ \\
		$M_3 = \{ \underline{3}, \underline{\{1,2\}}, \underline{M_1}\}$ \\
		geht nur bei endlichen Mengen oder bestimmten unendlichen Mengen, zum Beispiel: \\
		$\mathbb{N} = \{1,2,3,4,...\}$ Menge der natürlichen Zahlen \\
		$\mathbb{N}_0 = \{0, 1,2,3,4,...\}$ Menge der natürlichen Zahlen mit der Null \\
		$\mathbb{Z} = \{0,1,-1,2,-2,...\}$ Menge der ganzen Zahlen
		\item Charakterisierung ihrer Elemente: \\
		$A = \{x \mid x \text{ besitzt die Eigenschaft } E\}$, z.B.:\\
		$A = \{n \underbrace{\mid}_{\mathrlap{\text{sprich: \textit{''mit der Eigenschaft''}}}} n \in \mathbb{N} \text{ und n ist gerade}\}$\\
		$\quad = \{2,4,6,8,...\}$ \\
		$\quad = \{ x \mid \exists k \in \mathbb{N} \text{ mit } x = 2 \cdot k\} = \{2k \mid k \in \mathbb{N}\}$ \\
		
		Bsp: $\mathbb{Q} = \{\frac{a}{b} \mid a,b \in \mathbb{Z}, b \neq 0 \}$ Menge der rationalen Zahlen		
	\end{itemize}
\item[c)]
	Mit $\emptyset$ bezeichnen wir die Menge ohne Elemente (\underline{leere Menge})
\item[d)]
	Mit $\abs{A}$ bezeichnen wir die Anzahl der Elemente der Menge $A$ (\underline{Kardinalität} oder \underline{Mächtigkeit} von $A$), zum Beispiel: \\
	$\abs{\{1,a,*\}} = 3, \quad \abs{\emptyset} = 0, \quad \abs{\mathbb{N}} = \infty, \quad \abs{\{\mathbb{N}\}} = 1$
\item[e)]
	$A \cap B \underbrace{:=}_{\mathrlap{\text{wird definiert als}}} \{x \mid x \in A \wedge x \in B\}$ heißt \underline{Durchschnitt} oder \underline{Schnittmenge} von $A$ und $B$.
	
	Grafische Veranschaulichung: Venn-Diagramm (\attention gilt nicht als Beweis)
	
	\begin{venndiagram2sets}
	\fillACapB
	\end{venndiagram2sets}

	
\item[f)]
	$A \cup B :=\{x \mid x \in A \lor x \in B \}$ heißt \underline{Vereinigung} von $A$ und $B$.
		
	\begin{venndiagram2sets}
	\fillA \fillB
	\end{venndiagram2sets}
	
\item[Beispiele:]
	$A = \{1,2,3\}$, $B = \{2,3,4\}$, $C = \{4\}$\\ \\
		$A \cap B = \{2,3\}$,\\
		$A \cap C = \emptyset$,\\
		$B \cap C = \{4\} = C$,\\
		$A \cup B = \{1,2,3,4\}$
		
\item[g)]
	$A$ und $B$ heißen \underline{disjunkt}, falls gilt $A \cap B = \emptyset$
		
	\begin{venndiagram2sets}[overlap=-20]		
	\end{venndiagram2sets}
	
\item[h)]
	$A$ heißt \underline{Teilmenge} von $B$, $A \subseteq B$, falls gilt: \\
	$x \in A \Rightarrow x \in B$\\
	Oder in Worten: Jedes Element von $A$ ist auch Element von $B$.
	
	Dasselbe bedeutet die Notation\\
	$B \supseteq A$ \\
	($B$ ist Obermenge von $A$)
	
	Beispiel: $\{1,2\} \subseteq \{1,2,3\} \subseteq \mathbb{N} \subseteq \mathbb{N}_0 \subseteq \mathbb{Z} \subseteq \mathbb{R}$ (reelle Zahlen)
	
	Es gilt: $\emptyset \subseteq A$ für jede Menge $A$.

	\textbf{Achtung: } Unterschied $\subseteq, \in$ !\\
	Zum Beispiel: \\
	$A = \{1, \mathbb{N}\}$ (hier ist die Menge $\mathbb{N}$ ein Element von A, keine Teilmenge!)\\
	$1 \in A, \qquad \mathbb{N} \in A, \qquad \mathbb{N} \nsubseteq A, \qquad 2 \notin A, \qquad \{1\} \subseteq A$

% =============
% 28 Oktober 2015
% =============

\item[i)]
	Zwei Mengen A, B heißen \underline{gleich} $(A = B$, falls gilt: $A \subseteq B$ und $B \subseteq A$
	(also $x \in A \Rightarrow / \Leftarrow / \Leftrightarrow x \in B$.

	%TODO Diagramm

	Darin liegt ein Beweisprinzip: Man zeigt $A = B$, indem man zeigt:
	\begin{itemize}
	\item $x \in A \Rightarrow x \in B$
	\item $x \in B \Rightarrow x \in A$ (mehr später)
	\end{itemize}

	Beispiel:
	$A = {2, 3, 4}, \qquad B = \{ x \in \mathbb{N} \mid x > 1$ und $x < 5\}$
	$A = B$

\item[j)]
	$A \subsetneq B (A \subsetneqq B)$ bedeutet $A \subseteq B$, aber $A \neq B$.

	(d.h. $\exists x \in B$ mit $x \notin A$, aber $x \in B$)

	(A ist \underline{echte} Teilmenge von B.)

	%TODO Diagramm

\item[k)]
	Mit $P(A) := \{ B \mid \text{B ist eine Teilmenge von A}\} = \{B \mid B \subseteq A\}$
	bezeichnen wir die Menge aller (echten oder nicht echten) Teilmengen von A, die sogenannte \underline{Potenzmenge von A}.
	$(\emptyset \subseteq A \forall A, A \subseteq A \forall A)$

	Beispiel:

	$A = \{1,\}, P(A) = \{\emptyset, \{ \underbrace{1}_{A}\}\}$

	$B = \{1, 2\}, P(B) = \{ \emptyset, \{1\}, \{2\}, \{ \underbrace{1, 2}_{B}\}\}$

	$C = \{1, 2, 3\}, P(C) = ...$ (8 Elemente)

	$P(\emptyset) = \{ \emptyset \}$

	Was ist $P (P(A))$? \\
	$P(P(A)) = P(\{ \emptyset, \{ 1 \}\}) = \{ \emptyset, \{ \emptyset \}, \{1 \}, \{ \emptyset, \{ 1 \}\}$ %Klammern?

\item[l)]
	$A \backslash B := \{ x \mid x \in A$ und $x \notin B \}$ heißt die \underline{Differenz} (\textit{A ohne B}).

	Ist $A \subseteq X$ mit einer Obermenge $X$, so heißt $X \backslash A$ das \underline{Komplement} von $A$ (bezüglich $X$).
	Wir schreiben $A^C_X$ oder kurz $A^C$ (wenn X aus dem Kontext klar ist).

	%TODO Diagramme

\item[m)]
	$A \triangle B := (A \backslash B) \cup (B \backslash A)$ heißt die symmetrische Differenz von $A$ und $B$. %delta statt triangle?

	%TODO Diagramm

\end{description}

\subsection{Bemerkung}
Verallgemeinerung der Vereinigung und des Durchschnitts:

$A_1 \cap A_2 \cap ... \cap A_n = \{x \mid x \in A_1 \wedge x \in A_2 \wedge ... \wedge x \in A_n\}$

$$=: \bigcap_{i = 1}^{n} A_i$$

$A_1 \cup ... \cup A_n = \{x \mid x \in A_1 \lor  ... \lor x \in A_n\}$

$$=: \bigcup_{i = 1}^{n} A_i$$

Beziehungsweise noch allgemeiner:

Sei $S$ eine Menge von Mengen (\textit{System von Mengen})

$\cap A = \{ x \mid x \in A \forall A \in S\} $ \\
$ A \subset S$

$\cup A = \{ x \mid \exists A \in S$ mit $x \in A\} $ \\
$ A \in S$

\subsection{Definition}
Seien $A, B$ Mengen.

$A \underbrace{x}_{Kreuz} B := \{(a, b) \mid a \in A, b \in B\}$

Die Menge aller geordneten Paare, heißt \underline{kartesisches Produkt} von $A$ und $B$ (nach René Descartes, 1596 - 1650).

Dabei legen wir fest: $(a, b) = (a', b') (\text{mit } a, a' \in A, b, b' \in B):$ \\
$\Leftrightarrow a = a' \text{ und } b = b‘$.

Allgemein sei für Mengen $A_1, ... A_n (n \in \mathbb{N})$ \\
$A_1 x A_2 x ... x A_n := \{a_1, a_2, ..., a_n) \mid a_i \in A_i, \forall i = 1 ... n\}$ \\
die Menge aller \underline{geordneten n-Tupel} (mit analoger Gleichheitsdefinition).

$(n = 2: \text{Paare}, n = 3: \text{Tripel})$

Schreibweise: \\
$$A_1 \times ... \times A:n =: \bigtimes_{i=1}^{n} A_i$$

Ist eine der Mengen $A_1, ... A_n$ leer, setzen wir $A_1 \times ... \times A_n = \emptyset$.

Statt $A \times A$ schreiben wir auch $A^2$, statt $\underbrace{A \times ... \times A}_{n-Faktoren} = A^n$.


\subsection{Beispiel}
$A = \{1, 2, 3\}, B = \{3, 4\}$

$(1, 3) \in A \times B, \underbrace{(3, 1)}_{B \times A} \notin A \times B,$

$(\underbrace{3}_{B \times B}, \underbrace{3}_{A \times A}) \in A \times B\in B \times A$

$(1, 2) \in A \times B, \in A \times A$

$A \times B = \{(1, 3), (1, 4), (2, 3), (2, 4), (3, 3), (3, 4)\}$

$B \times A = ...$

$B \times B = B^2 = \{(3, 3), (3, 4), (4, 3), (4, 4)\}$

% ==============
% 2 November 2015
% ==============

\subsection{Satz (Rechenregeln für Mengen)}

Seien $A$, $B$, $C$, $X$ Mengen. Dann gilt:
\begin{itemize}
	\item[a)]
		$A \cup B = B \cup A$ \\
		$A \cap B = B \cap A$ \\
		(Kommutativgesetz)
		
	\item[b)]
		$(A \cup B) \cup C = A \cup (B \cup C)$ \\	
		$(A \cap B) \cap C = A \cap (B \cap C)$ \\
		(Assoziativgesetz)
		
	\item[c)]
		$(A \cup B) \cap C = (A \cap C) \cup (B \cap C)$ \\
		$(A \cap B) \cup C = (A \cup C) \cap (B \cup C)$ \\
		(Disbributivgesetz)
		
	\item[d)]
		$A,B \subseteq X$, dann \\
		$(A \cap B)^C_X = A^C_X \cup B^C_X$ \\
		$(A \cup B)^C_X = A^C_X \cap B^C_X$ \\
		(Regeln von DeMorgan)
		
	\item[e)]
		$A \subseteq X$, dann $(A^C_X)^C_X = A$
		
	\item[f)]
		$A \Delta B = (A \cup B) \setminus (A \cap B)$
		
		$(=\{x \mid x \in A \oplus x \in B\})$
			
		\begin{venndiagram2sets}
		\fillANotB \fillBNotA
		\end{venndiagram2sets}
		
	\item[g)]
		$A \cap B = A$ genau dann, wenn $A \subseteq B$ \\
		$(A \cap B) = A \quad \Leftrightarrow \quad A \subseteq B)$
		
	\item[h)]
		$A \cup B = A \quad \Leftrightarrow \quad B \subseteq A$
\end{itemize}

\subsubsection*{Beweis}

\begin{itemize}
	\item[a)]
		$A \cup B = \{x \mid x \in A \lor x \in B\}$ \\
		$\qquad \underset{\mathllap{\text{Kommutativgesetz 1.9 b)}}}{=} \{x \mid x \in B \lor x \in A\} = B \cup A$ \\		
		\hfill \\		
		$A \cap B$ analog
		
	\item[b), c)]
		Übung, wie a) \\
		benutze Assoziativgesetz (1.9 c) ) bzw. Distributivgesetz (1.9 d) ) für logische Äquivalenzen.
		
	\item[d)]
		$(A \cap B)^C_X$ \\
		$ = \{x \mid x \in X \setminus (A \cap B) \}$ \\
		$ = \{x \mid x \in X \land (x \notin (A \cap B)) \}$ \\
		$ = \{x \mid x \in X \land \neg (x \in (A \cap B)) \}$ \\
		$ = \{x \mid x \in X \land \neg (x \in A \land x \in B) \}$ \\
		$ \underset{\mathllap{\text{De Morgan 1.9 e)}}}{=} \{x \mid x \in X \land (x \notin A \lor x \notin B)\}$ \\
		$ = \{x \mid ((x \in X) \land (x \notin A)) \lor ((x \in X) \land (x \notin B)) \}$ \\
		$ = A^C_X \cup B^C_X$
		
		2. Regel analog
		
	\item[e)]
		ähnlich
	\item[f) g) h)]
		später
\end{itemize}

%%%%%%%%%%%%%%%%%%%%%%%%%%%%%%%
% Kapitel 3: Beweismethoden
%%%%%%%%%%%%%%%%%%%%%%%%%%%%%%%
\section{Beweismethoden}

Ein mathematischer \underline{Beweis} ist die Herleitung der Wahrheit (oder Falschheit) einer Aussage aus einer Menge von \underline{Axiomen} (nicht beweisbare Grundtatsachen) oder bereits bewiesenen Aussagen nmittels logischen Folgerungen.

Bewiesene Aussagen werden \underline{Sätze} genannt.

\hfill

\underline{Lemma} - Hilfssatz, der nur als Grundlage für wichtigeren Satz formuliert und bewiesen wird.

\underline{Theorem} - wichtiger Satz

\underline{Korollar} - einfache Folgerung aus Satz, z.B. Spezialfall

\underline{Definition} - Benennung/Bestimmung eines Begriffs/Symbols

$\square$ - Zeichen für Beweisende ($\blacksquare$, q.e.d., wzbw...)

\hfill

Mathematische Sätze haben oft die Form:

Wenn $V$ (Voraussetzung) gilt, dann gilt auch $B$ (Behauptung)

($V$, $B$: Aussagen), kurz: $V \Rightarrow B$

Zu zeigen ist also, dass $V \Rightarrow B$ eine wahre Aussage ist.

\subsection{Direkter Beweis}

Gehe davon aus, dass $V$ wahr ist, folgere daraus, dass $B$ wahr ist.

[\space\space Sei $V$ wahr, $\Rightarrow$ ... \\
\text{\qquad\qquad\qquad} $\Rightarrow$ ... \\
\text{\qquad\qquad\qquad} $\Rightarrow$ ... \\
\text{\qquad\qquad\qquad} \space $\vdots$ \\
\text{\qquad\qquad\qquad} $\Rightarrow$ $B$ ist wahr\space\space]

Beispiel: $\underbrace{\text{Sei $n \in \mathbb{N}$. Ist $n$ gerade}}_{V}$, $\underbrace{\text{so ist auch $n^2$ gerade}}_{B}.$

\underline{Beweis:} Sei $n \in \mathbb{N}$ gerade. \hfill // V ist wahr
$\Rightarrow n = 2 \cdot k$ für ein $k \in \mathbb{N}$ \\
\text{\qquad\qquad} ($\exists k \in \mathbb{N}$ mit $n = 2 \cdot k$) \\
$\Rightarrow n^2 = (2 \cdot k)^2 = 4 \cdot k^2 = 2 \cdot (2k^2)$ \\
$\Rightarrow n^2$ ist gerade. \hfill // B ist wahr

\hfill $\square$

\subsection{Beweis durch Kontraposition}

vgl. Satz 1.9 f) \qquad $A \Rightarrow B \quad \equiv \quad \neg B \Rightarrow \neg A$

Statt $V \Rightarrow B$ zu zeigen, können wir also auch $\neg B \Rightarrow \neg V$ zeigen.


[ Es gelte $\neg B \Rightarrow$ ... \\
\text{\qquad\qquad\qquad} $\Rightarrow$ ... \\
\text{\qquad\qquad\qquad} $\Rightarrow$ ... \\
\text{\qquad\qquad\qquad} \space $\vdots$ \\
\text{\qquad\qquad\qquad} $\Rightarrow$ es gilt $\neg V$ ]

\hfill

\underline{Beispiel:} Sei $n \in \mathbb{N}$.

$\underbrace{\text{Ist $n^2$ gerade}}_{V}$, $\underbrace{\text{so ist auch $n$ gerade}}_{B}$.

\hfill

\underline{Beweis durch Kontraposition:}

Sei $n$ ungerade. \hfill // $\neg B$ gilt.

$\Rightarrow n = 2k + 1$ für ein $k \in \mathbb{N}_0$ \\
$\Rightarrow n^2 = (2k+1)^2 = 4k^2+4k+1 = \underbrace{\underbrace{2(2k^2+2k)}_{\text{gerade}}+1}_{\text{ungerade}}$ \\
$\Rightarrow n^2$ ist ungerade. \hfill // $\neg V$ gilt.

\hfill $\square$

\subsection{Beweis durch Widerspruch, indirekter Beweis}

Zu zeigen ist Aussage $A$. Wir gehen davon aus, dass $A$ \underline{nicht} gelte ($\neg A$ ist wahr) und folgern durch logische Schlüsse eine zweite Aussage $B$, von der wir wissen, dass sie falsch ist. Wenn alle logischen Schlüsse korrekt waren, muss also $\neg A$ falsch gewesen sein, also $A$ wahr.

( $((\neg A \Rightarrow B) \land (\neg B)) \Rightarrow A$ ist Tautologie)

% ==============
% 4 November 2015
% ==============

\textbf{Beispiel:} [Euklid] $\sqrt{2} \notin \mathbb{Q}$

\underline{Beweis:} Wir nehmen an, dass die Aussage falsch ist, also $\sqrt{2} \in \mathbb{Q}$ gilt,
das heißt $\sqrt{2} = \frac{p}{q}$ mit p. q. $\in \mathbb{Z} (q \neq 0)$ teilerfremd (vollständig gekürzter Bruch)

$\Rightarrow 2 = \frac{p^2}{q^2}$

$\Rightarrow p^2 = 2q^2$, also ist $p^2$ gerade, damit aber auch p gerade (Beispiel in 3.2), also $p = 2 * r$ mit $r \in \mathbb{Z}$.

$\Rightarrow p^2 = (2r)^2 = 2q^2$ \\
$\Rightarrow 4r^2 = 2q^2$ \\
$\Rightarrow \underline{2r^2 = q^2}$ \\
$\Rightarrow q^2$ gerade \\
$\Rightarrow q$ gerade

Also: $p$ gerade, $q$ gerade, Widerspruch zu $p, q$ teilerfremd.

Also war die Annahme falsch, es muss $\sqrt{2} \notin \mathbb{Q}$ gelten. $\square$

\subsection{Vollständige Induktion}
Eine Methode, um Aussagen über natürliche Zahlen zu beweisen.

\textbf{Beispiel:} Gauß

$ 1 + 2 + ... + 100 = ?$

\begin{tabular}{c c c c c }
1 & 2 & 3 & ... & 50 \\
+ 100 & 99 & 98 & ... & 51 \\
\hline
101 & 101 & 101 & ... & 101 \\
\end{tabular}

$50 * 101 = 5050$

$(= \frac{100}{2} * 101)$

\underline{Allgemein:} \\
$ 1 + 2 + 3 + ... + n \underbrace{=}_{Vermutung} \frac{n (n+1)}{2}$ \\
$(n \in \mathbb{N})$

\subsubsection{Prinzip der vollständigen Induktion}
Sei $n_0 \in \mathbb{N}$ fest vorgegeben (oft $n_0 = 1)$. \\
Für jedes $n \geq n_0, n \in \mathbb{N}$, sei $A(n)$ eine Aussage, die von $n$ abhängt.

Es gelte:
\begin{enumerate}
\item $A(n_0)$ ist wahr (\textit{Induktionsanfang})
\item $\forall n \in \mathbb{N}, n \geq n_0$:
$\underbrace{\text{Ist} A(n) \text{wahr,}}_{Induktionsvorraussetzung} \underbrace{\text{so ist} A(n+1) \text{wahr}}_{Induktionsbehauptung}.$ (\textit{Induktionsschritt})
\end{enumerate}

Dann ist die Aussage $A(n)$ für alle $n \geq n_0$ wahr. (\textit{Dominoprinzip})

(\underline{Bemerkung}: gilt auch für $\mathbb{N}_0$ ($n_0 = 0$ auch möglich) und für $n_0 \in \mathbb{Z}$, Behauptung gilt dann für alle $n \in \mathbb{Z}$ mit $n \geq n_0$).

\underline{Beispiel}: 

\begin{description}

\item[a) Kleiner Gauß]
$1 + 2 + ... + n = \frac{n(n+1)}{2} \forall n \in \mathbb{N}$

\underline{Beweis}:

$A(n) : 1 + 2 + ... + n = \frac{n(n+1)}{2}$

\begin{itemize}
\item Induktionsanfang $(n = 1): (A(1): 1 = \frac{1*(1+1)}{2}$
\item Induktionsschritt:

Induktionsvorraussetzung: sei $n \geq 1$. Es gelte $A(n)$, d.h. $1+ ... +n = \frac{n(n+1)}{2}$

Induktionsbehauptung: Es gilt $A(n+1)$, d.h. $1+ ... +n + (n+1) = \frac{(n+1) (n+1 + 1)}{2}$

Beweis: $\underbrace{1 + 2 + ... + n}_{} + (n+1) \underbrace{=}_{Ind.vor.} \underbrace{\frac{n(n+1)}{2}} + (n+1)$

$\qquad\qquad\qquad\qquad\qquad\qquad\qquad$ \space
$ = \frac{n^2 + n + 2n + 2}{2}$

$\qquad\qquad\qquad\qquad\qquad\qquad\qquad$ \space
$=\frac{(n+1)(n+2)}{2}$

$A(n+1)$ \hfill $\square$

\end{itemize}

\item[b)]

$A(n): 2^n \geq n \forall n \in \mathbb{N}$
\begin{itemize}
\item Induktionsanfang: $(n = 1 ) : A(1)$ gilt: $2^1 \geq 1$
\item Induktionsschritt:

Induktionsvorraussetzung: Sei $n \geq 1$. Es gelte $A(n)$, d.h. $2^n \geq n$

Induktionsbehauptung: (Zu zeigen!): Es gilt $A(n+1)$, d.h. $2^{2+1} \geq n+1$.

Beweis: $2^{n+1} = 2*2^n \underbrace{\geq}_{Ind.vor.} 2 * n$

$\qquad\qquad\qquad\qquad\qquad$
$= n + n$

$\qquad\qquad\qquad\qquad\qquad$
$\geq n + 1$,

$\qquad\qquad$
also \qquad $2^{n+1} \geq n+1$ \hfill $\square$
\end{itemize}

\end{description}

\subsubsection{Bemerkung}
Für Formeln wie in Beispiel 3.4.1a) benutzen wir das \textit{Summenzeichen} $\Sigma$ (sigma, großes griechisches S)

$\displaystyle\sum_{k = 1}^{n} k = \frac{n(n+1)}{2}$
$1 + 2 + 3 + ... +n$
$k = 1 k = 2 k = 3 k = n$ %Tabelle

weitere Bsp:

$\sum_{k = 1}^{n} 2k = 2*1 + 2*2 + ... 2*n$
$\sum_{k=4}^{n} 2k = 2*4 + 2*5 + .... 2*n$

$\sum_{k=1}^{3} 7 = 7 + 7 + 7 = 21$ %unten drunter k = 1 k= 2 k = 3

allg. $\sum_{k=m}^{n} a_k = a_m + a_{m+1} + a_n$
$(a_m, a_{m+1}, ... a–n \in \mathbb{R})$

h heißt Summationszeichen

$\sum_{k=m}^{n} a_k = \sum_{i = m}^{n} a_i$

Schreibweisen:

$\displaystyle\sum_{k = 1}^{n} a_k, \sum_{k = 1}^{n} a_k, \sum_{k \in \mathbb{N}} a_k, \sum_{k=1, k \neq 2}^{4} a_k = a_1 + a_3 + a_4$

Für $n < m$ setzt man

$\sum_{k=m}^{n} a_k = 0 (\textit{leere Summe})$, z.B. $\sum_{k=7}^{3} k = 0$ \\

% ==============
% 9 November 2015
% ==============

\textbf{Produktzeichen $\Pi$}

$\displaystyle\prod_{k=m}^{n} a_k = a_m * a_{m+1} ... a_n,$ \\
für $n < m$ setze $\displaystyle\prod_{k=m}^{n} a_k = 1$

\underline{Rechenregeln für Summen} (zu beweisen z.B. durch vollständige Induktion)

\begin{description}
	\item[a)] \hfill

	$\sum_{k=m}^{n} a = (n - m+ 1) * a$

	$(\sum_{k=3}^{5} a = a + a + a = (5-3+1)*a)$

	\item[b)] \hfill

	$\sum_{k=m}^{n} (c * a_k) = c * \sum_{k=m}^{n} a_k$

	\item[c) Indexverschiebung] \hfill

	$\sum_{k=m}^{n} a_k = a_m + a_{m+1} + ... a_n$ \\
	$\qquad = a_{(m + e) - e} + a_{(m+1+e)-e} + ... + a_{(n+e)-e}$ \\
	neuer Summationsindex $j := k +e$ \\
	(k durchläuft Werte: $m, m+1 ..., n$, \\
	j durchläuft Werte: $m + e, m+1+e, ... n+e)$

	also gilt
	$\sum_{k=m}^{n} a_k = \sum_{j = m+e}^{n+e} a_{j-e}$

	(Beispiel:
	$\sum_{k=0}^{5} a_k * x^{k+2} = \sum_{j = 2}^{7} a_{j-2} * x^j)$

	\item[d) Addition von Summen gleicher Länge] \hfill

	$\sum_{k=m}^{n} (a_k + b_k) = \sum_{k=m}^{n} a_k + \sum_{k=m}^{n} b_k$

	\item[e) Aufspalten] \hfill

	$\sum_{k=m}^{n} a_k = \sum_{k=m}^{l} a_k + \sum_{k=l+1}^{n} a_k$ für  $m < l < n$

	\item[f) Teleskopsumme] \hfill

	$\sum_{k=m}^{n} (a_k - a_{k+1}) = a_m - a_{n+1}$

	$\sum_{k=m}^{n} (a_k - a_{k+1} = (a_m - a_{m+1} + (a_{m+1} - a_{m+2} + (a_{m+2} ... ) + (a_n - a{n+1}))$ % Durchgestrichen

	\item[g) Doppelsummen] \hfill

	$\sum_{i=1}^{n} \sum_{j=1}^{m} a_{ij}$
	$= \sum_{i=1}^{n} (a_{i1} + a_{i2} + ... + a_{im}$
	$= a_{11} + a_{12} + ... + a_{1m}$ %TODO Klammer?

	$+ a_{21} + a_{22} + a_{2m}$

\end{description}

\subsubsection{Verschärftes Induktionsprinzip}

$A(n), n_0$ wie in 3.4.1

Es gelte:
\begin{itemize}
	\item[(1)] $A(n_0)$ ist wahr
	\item[(2)] $\forall n \geq n_0:$ \\
	Sind $A(n_0), \quad ... \quad , A(n)$ wahr, so ist $A(n+1)$ wahr.
	
	(d.h. $A(n_0) \land A(n_0+1) \land ... \land A(n) \Rightarrow A(n+1)$)
\end{itemize}
Dann ist A(n) wahr für \underline{alle} $n \in \mathbb{N}, n \geq n_0$

\hfill

\underline{Beispiel: } $A(n)$: Jede natürliche Zahl $n > 1$ ist Primzahl oder Produkt von Primzahlen.

\underline{Beweis:}

\underline{Induktionsanfang:} ($n_0=2$). $n=2$ ist Primzahl \checkmark

\underline{Induktionsschritt:} Sei $n \geq n_0 \qquad (n \geq 2)$

\underline{$\bullet$ Induktionsvoraussetzung:}

Aussage gilt für $2,3,4,...,n$

($A(2),A(3),A(4),...,A(n)$ wahr)

\underline{$\bullet$ Induktionsbehauptung:}

$A(n+1)$ gilt, d.h. $n+1$ ist Primzahl oder Produkt von Primzahlen.

Beweis:

\begin{itemize}
\item falls $n+1$ Primzahl, so gilt $A(n+1)$
\item falls $n+1$ keine Primzahl, dann ist $n+1 = k \cdot l$, für $k,l \in \mathbb{N}$, \\
$1 < k < n+1, 1 < l < n+1$ ($k=l$ möglich).

Nach Induktionsvoraussetzung:

Aussage gilt für $k$ und $l$ $\Rightarrow$ $n+1$ ist Produkt von Primzahlen. \\
($A(n+1)$ ist wahr) \hfill $\square$
\end{itemize}


\subsection{Schubfachprinzip}

\subsubsection{Idee} In einem Schrank befinden sich $n$ verschiedene Paar Schuhe. Wie viele Schuhe muss man maximal herausziehen, bis man sicher ein zusammenpassendes Paar hat?

(Antwort: $n+1$)

\subsubsection{Satz} (Schubfachprinzip, engl.: {\it pigeon hole principle})

Seien $k,n \in \mathbb{N}$.

Verteilt man $n$ Objekte auf $k$ Fächer, so gibt es ein Fach, das mindestens $\ceil{\frac{n}{k}}$ Objekte enthält.

(Dabei bezeichnet $\ceil{x}$ die kleinste ganze Zahl $z$ mit $x \leq z$.)

\underline{Beweis} (durch Kontraposition):

( $\underbrace{n \text{ Objekte, } k \text{ Fächer}}_{A} \Rightarrow \underbrace{\exists \text{ Fach mit mind. } \ceil{\frac{n}{k}} \text{ Objekten}}_{B}$

statt $A \Rightarrow B$ zeige $\neg B \Rightarrow \neg A$ )

\begin{itemize}
\item[$(\neg B)$]
	Jedes Fach enthalte höchstens $\ceil{\frac{n}{k}}-1$ Objekte.
	
	Dann ist die Gesamtzahl von Objekten höchstens
	
	$$k \cdot \underbrace{(\ceil{\frac{n}{k}}-1)}_{< \frac{n}{k}} < k \cdot \frac{n}{k} = n$$

\item[$(\neg A)$]
	es gibt also \underline{weniger} als $n$ Objekte
	\hfill $\square$

\end{itemize}

% ==============
% 11 November 2015
% ==============

\subsubsection{Beispiel}

\begin{description}
\item[a)]
Wieviele Menschen müssen auf einer Party sein, damit \underline{sicher} 2 am selben Tag Geburtstag haben?

367

\item[b)]
Auf jeder Party mit mindestens 2 Gästen gibt es 2 Personen, die dieselbe Anzahl \underline{Freunde} auf der Party haben.

Beweis: Sei $n$ die Anzahl der Partygäste. Jeder Gast kann mit $0, 1, 2, ..., n-1$ Gästen befreundet sein ($n$ Möglichkeiten).

Aber: Es kann nicht sein, dass ein Gast $0$ Freunde hat und gleichzeitig ein Gast $n-1$ (=alle) Freunde hat.

$\Rightarrow$ Es gibt $n-1$ mögliche Werte für die Anzahl der Freunde, entspricht $n-1$ Fächern.

Jeder der $n$ Gäste trägt sich in ein Fach ein \\
$\Rightarrow$ mindestens $2$ Gäste sind im selben Fach. \hfill $\square$

\item[c)]
In Berlin gibt es mindestens 2 Personen, die genau dieselbe Anzahl Haare auf dem Kopf haben.

Beweis: Anzahl Haare im Durchschnitt:

\begin{tabular}{ c c }
blond & 150.000 \\
braun & 110.000 \\
schwarz & 100.000 \\
rot & 90.000
\end{tabular}

zur Sicherheit: maximal 1 Millionen Haare möglich \\
entspricht 1 Mio Fächer.

Anzahl Einwohner in Berlin: 3,5 Millionen $\Rightarrow$ Behauptung 3.5.2 \hfill $\square$

\end{description}

\subsection{Weitere Beweistechniken (Werkzeugkiste)}

\begin{description}
\item[a)]
Wichtigste Technik: Ersetzen eines mathematischen Begriffs durch seine Definition (und umgekehrt).
$A( \subset B = \{x \mid x \in A \lor x \in B\})$

\item[b)]
Aussagen der Form $\forall a \in S$ gilt $P(a)$: \\
beginne mit: Sei $a \in S$, zeige $P(a)$.

\item[c)]
Aussage der Form $\exists a \in S$ mit $P(a)$ \\
oft: finde/gebe konkretes Element $a$ an, für dass $P(a)$ gilt.

\item[d)]
Gleichheit von Mengen zeigt man oft mittels Inklusion (vgl. Definition 2.1(i))

Zu zeigen: $A = B$ ($A, B$ Mengen) \\
zeige: $A \subseteq B$ (Sei $a \in A \Rightarrow ... \Rightarrow ... \Rightarrow a \in B$) 2.1 (i)) \\
und $B \subseteq A$ (Sei $b \in B \Rightarrow ... \Rightarrow ... \Rightarrow b \in A$)

\textit{$\subseteq$} ...\\
\textit{$\supseteq$} ... \\

\underline{Beispiel:} 2.5f)

$A \triangle B = (A \cup B) \backslash (A \cap B)$

Beweis: \\
\textit{$\subseteq$} Sei $x \in A \triangle B = (A \backslash B) \cup (B \backslash A)$

\begin{itemize}
\item[1. Fall]: \hfill \\
$x \in A \backslash B$, dann gilt $x \in A$, also $x \in A \cup B$

Außerdem $x \notin B$, also gilt auch $x \notin A \cap B$

$\Rightarrow x \in (A \cup B) \backslash (A \cap B)$

\item[2.Fall] \hfill \\
Ist $ x \in B \backslash A$, so argumentiere analog.
\end{itemize}

\textit{$\supseteq$} sei $x \in (A \cup B) \backslash (A \cap B)$ \\
$\Rightarrow x \in A$ oder $x \in B$.

\begin{itemize}
\item[1.Fall] \hfill \\
$x \in A$, so ist $x \notin B$, da $x \notin A \cap B$ \\
$\Rightarrow x \in A \backslash B \subseteq (A \backslash B) \cup (B \backslash A)$ \\
$ = A \triangle B$, \\
d.h. $x \in A \triangle B$.

\item[2.Fall] (1. Fall analog) \hfill \\
$x \in B$, so $x \notin A$, da $x \notin A \cap B$ \\
$\Rightarrow x \in B \backslash A \subseteq A \triangle B$ \\
Also $x \in A \triangle B$
\end{itemize}

\item[e)]
Äquivalenzen $(A \Leftrightarrow B, A, B$ Aussagen) werden meist in 2 Schritten bewiesen:

\textit{Hinrichtung} zeigt $A \Rightarrow B$, \\
\textit{Rückrichtung} zeigt $B \Rightarrow A$.

\textit{$\Rightarrow$: ...} \\
\textit{$\Leftarrow$: ...}

(oft auch eine von beiden mittels Kontraposition)

\underline{Beispiel:} 2.5g) $A \cap B = A \Leftrightarrow A \subseteq B$

Beweis: \\
\textit{$\Rightarrow$}: Sei $ A \cap B = A$. Dann ist $A = A \cap B \subseteq B$ \\
\textit{$\Leftarrow$}: Sei $A \subseteq B$. Dann ist $A \subseteq A$ und $A \subseteq B$, \\
also ist $A \subseteq A \cap B$ \\
außerdem $A \cap B \subseteq A$

$\Rightarrow A = A \cap B$ \hfill $\square$

2.5h) analog.

\item[f)]
Äquivalenzen der Form: \\
Sei ... . Dann sind folgende Aussagen äquivalent:

\begin{itemize}
\item a) ...
\item b) ...
\item c) ..
\item d) ...
\end{itemize}

Zeigt man durch \textit{Ringschluss}: \\
Zeige $a) \Rightarrow b) \Rightarrow c) \Rightarrow d) \Rightarrow a)$ \\
(oder andere Reihenfolge, soll \textit{Ring} geben.)

\end{description}

% ==============
% 16 November 2015
% ==============

%%%%%%%%%%%%%%%%%%%%%%%%%%%%%%%
% Kapitel 4: Abbildungen
%%%%%%%%%%%%%%%%%%%%%%%%%%%%%%%
\section{Abbildungen}

\subsection{Definition}

\begin{itemize}
\item[a)] Eine \underline{Abbildung} (oder \underline{Funktion})
$$f: A \rightarrow B$$ besteht aus
	\begin{itemize}
	\item zwei nicht-leeren Mengen:\\
		$A$, dem \underline{Definitionsbereich} von f \\
		$B$, dem \underline{Bildbereich} von f
	\item und einer Zuordnungsvorschrift, die \underline{jedem} Element 
	$a \in A$ \underline{genau} \underline{ein} Element $b \in B$ zuordnet
	\end{itemize}
	
Wir schreiben dann $b = f(a)$, nennen $b$ das \underline{Bild} oder den \underline{Funktionswert} von $a$ (unter $f$), und $a$ (ein) \underline{Urbild} von $b$ (unter $f$).

Notation:
$$f: A \rightarrow B$$
$$\qquad\quad a \mapsto f(a)$$

\item[b)] Die Menge $G_f := \{(a,f(a)) \mid a \in A\} \subseteq A \times B$ heißt der \underline{Graph} von $f$.

\end{itemize}

\subsection{Beispiele} Siehe Folien!

\subsection{Beispiele}

\begin{itemize}
\item[a)] $A$ Menge

	$id_A : A \rightarrow A$ \\
	$\text{\qquad\space} x \mapsto x$
	
	identische Abbildung
	
\item[b)] 
		$\text{\quad} f : \mathbb{R} \rightarrow \mathbb{R}$ \\
		$\text{\space\qquad\space} x \mapsto x^2$ ist Abbildung (aus der Schule bekannt als $f(x)=x^2$)
		%TODO Graph von x^2
		
\item[c)] $\land$ kann als Abbildung aufgefasst werden, $+$ ebenso:

$\land : \{0,1\} \times \{0,1\} \rightarrow \{0,1\}$ \\
$\text{\qquad\qquad\quad} (A,B) \mapsto A \land B$

$+ : \mathbb{R} \times \mathbb{R} \rightarrow \mathbb{R}$ \\
$\text{\qquad}(a,b) \mapsto a+b$
\end{itemize}

Allgemein bezeichnet man eine Abbildung $\{0,1\}^n \rightarrow \{0,1\}^m$ ($n,m \in \mathbb{N}$) als boolesche Funktion.

\subsection{Definition}

Zwei Abbildungen $f: A \rightarrow B$, $g: C \rightarrow D$ heißen \underline{gleich} (in Zeichen: $f=g$), wenn:
\begin{itemize}
\item $A=C$
\item $B=D$
\item $f(a)=g(a)$
\end{itemize}
$\forall a \in A (=C)$

\subsection{Beispiel}

$f: \{0,1\} \rightarrow \{0,1\}, x \mapsto x$ \\
$g: \{0,1\} \rightarrow \{0,1\}, x \mapsto x^2$

$f=g$

\subsection{Definition}

Sei $f: A \rightarrow B$, seien $A_1 \subseteq A, B_1 \subseteq B$ Teilmengen.

Dann heißt

\begin{itemize}
\item[a)] $f(A_1) := \{f(a) \mid a \in A_1\} \subseteq B$ das \underline{Bild} von $A_1$ (unter $f$) (Bildmenge).

(Beispiel: $f: \mathbb{N} \rightarrow \mathbb{N}$ \\
$\text{\qquad\qquad\qquad} x \mapsto 2x$ \\
$\text{\qquad\qquad} A_1 = \{1,3\}$ \\
$\text{\space\qquad\space} f(A_1) = \{f(1), f(3)\} = \{2,6\}$ )

\item[b)] $f^{-1}(B_1) := \{a \in A \mid f(a) \in B_1\} \subseteq A$ \\
das \underline{Urbild von $B_1$} (unter $f$).

(Beispiel oben: $B_1 = \{8,14,100\}, f^{-1}(B_1) = \{4,7,50\}$ \\
$\text{\qquad\qquad\qquad\quad} B_2 = \{3\}, f^{-1}(B_2) = \emptyset$ )

\item[c)] $f$ \underline{surjektiv}, falls gilt: $f(a) = B$

(d.h. $\forall b \in B \exists a \in A : f(a) = b$ )

{\color{orange} [ alle Elemente von B werden getroffen ] }

\item[d)] $f$ \underline{injektiv}, falls gilt:

$\forall a_1, a_2 \in A$ mit $a_1 \neq a_2$ gilt $f(a_1) \neq f(a_2)$

(äquivalent: $f(a_1) = f(a_2) \Rightarrow a_1 = a_2$ )

{\color{orange} [ kein Element von B wird doppelt getroffen ] }

\item[e)] $f$ \underline{bijektiv}, falls $f$ surjektiv und injektiv ($f$ ist Bijektion).

{\color{orange} [ jedes Element wird genau einmal getroffen ] }

\end{itemize}

\subsection{Beispiele} siehe Folien


\begin{itemize}
\item[a)] $f$ aus Beispiel in 4.6 a) ist injektiv, aber nicht surjektiv:

$f(\mathbb{N})$ ist Menge der geraden natürlichen Zahlen, nicht $\mathbb{N}$.

\item[b)] $f: \mathbb{R} \rightarrow \mathbb{R}$ \\
$\text{\space\quad\space} x \mapsto x^2$

nicht surjektiv:

$f(\mathbb{R}) = \mathbb{R}^+_0 = \{x \in \mathbb{R} \mid x \geq 0 \} \neq \mathbb{R}$

nicht injektiv:

$f(1) = f(-1) = 1$ \\
$f(2) = f(-2) = 4$

\hfill

$g: \mathbb{R}^+_0 \rightarrow \mathbb{R}^+_0$ \\
$\text{\qquad} x \mapsto x^2$

injektiv, surjektiv, bijektiv

\item[c)] $f: \mathbb{R} \rightarrow \mathbb{R}$ \\
	$x \mapsto 2x+1$
	
	ist surjektiv:
	
	Sei $y \in \mathbb{R}$. Zeige: $\exists x \in \mathbb{R}$ mit $y = 2x+1$ (vgl. 3.6 b) )
	
	Wähle $x = \frac{y-1}{2}$
	
	$f$ ist injektiv:
	
	angenommen, es gibt $x_1,x_2 \in \mathbb{R}$ \\
	mit $f(x_1) = f(x_2)$, d.h. \\
	$2x_1+1 = 2x_2+1$, \\
	dann folgt $x_1 = x_2$. \qquad $\circledast$	
\end{itemize}

\end{document}