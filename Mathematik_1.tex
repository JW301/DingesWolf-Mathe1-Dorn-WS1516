% !TEX TS-program = pdflatex
% !TEX encoding = UTF-8 Unicode

% \documentclass[12pt]{article} % use larger type; default would be 10pt
\documentclass[a4paper, 12pt, twoside] {article}

\usepackage[utf8]{inputenc} % set input encoding (not needed with XeLaTeX)
\usepackage[ngerman]{babel}

\usepackage{amssymb} % math stuff
\usepackage{amsmath} % math stuff
\usepackage{multicol}

\usepackage{geometry} 
\usepackage{graphicx}

\usepackage[parfill]{parskip} % Activate to begin paragraphs with an empty line rather than an indent

%%% PACKAGES
\usepackage{booktabs} % for much better looking tables
\usepackage{array} % for better arrays (eg matrices) in maths
\usepackage{paralist} % very flexible & customisable lists (eg. enumerate/itemize, etc.)
\usepackage{verbatim} % adds environment for commenting out blocks of text & for better verbatim
\usepackage{subfig} % make it possible to include more than one captioned figure/table in a single float
% These packages are all incorporated in the memoir class to one degree or another...

%%% HEADERS & FOOTERS
\usepackage{fancyhdr} % This should be set AFTER setting up the page geometry
\pagestyle{fancy} % options: empty , plain , fancy
\renewcommand{\headrulewidth}{0pt} % customise the layout...
\lhead{}\chead{}\rhead{}
\lfoot{}\cfoot{\thepage}\rfoot{}

%%% SECTION TITLE APPEARANCE
\usepackage{sectsty}
\allsectionsfont{\sffamily\mdseries\upshape} % (See the fntguide.pdf for font help)
% (This matches ConTeXt defaults)

%%% ToC (table of contents) APPEARANCE
\usepackage[nottoc,notlof,notlot]{tocbibind} % Put the bibliography in the ToC
\usepackage[titles,subfigure]{tocloft} % Alter the style of the Table of Contents
\renewcommand{\cftsecfont}{\rmfamily\mdseries\upshape}
\renewcommand{\cftsecpagefont}{\rmfamily\mdseries\upshape} % No bold!

%%% END Article customizations

%%% CONTENT starts here

\title{Mathematik I WS 15/16}
\author{Thomas Dinges \thanks{thomas.dinges@student.uni-tuebingen.de}}

\begin{document}
\maketitle

\vfill
\thanks{Inoffizielles Skript für die Vorlesung Mathematik I im WS 15/16, bei Britta Dorn. Alle Angaben ohne Gewähr. Fehler können gerne via E-Mail gemeldet werden.}

\newpage

\tableofcontents

\newpage

\section{Logik}

\subsection*{Aussagenlogik}
Eine \textbf{logische Aussage} ist ein Satz, der entweder wahr oder falsch (also nie beides zugleich) ist. 
Wahre Aussagen haben den Wahrheitswert 1 (auch wahr, w, true, t), falsche den Wert 0 (auch falsch, f, false).

Notation: Aussagenvariablen $A, B, C, ... A_1, A_2$.

Beispiele:
\begin{itemize}
\item 2 ist eine gerade Zahl (1)
\item Heute ist Montag (1)
\item 2 ist eine Primzahl (1)
\item 12 ist eine Primzahl (0)
\item Es gibt unendlich viele Primzahlen (1)
\item Es gibt unendlich viele Primzahlzwillinge (Aussage, aber unbekannt, ob 1 oder 0)
\item 7 (keine Aussage)
\item Ist 173 eine Primzahl? (keine Aussage)
\end{itemize}

Aus einfachen Aussagen kann man durch logische Verknüpfungen (\textbf{Junktoren}, z.B. und, oder, ...) kompliziertere bilden. Diese werden Ausdrücke genannt (auch Aussagen sind Ausdrücke). 
Durch sogenannte \textbf{Wahrheitstafeln} gibt man an, wie der Wahrheitswert der zusammengesetzten Aussage durch die Werte der Teilaussagen bedingt ist. Im folgenden seien $A, B$ Aussagen. 

Die wichtigsten Junktoren:

\subsection{Negation}
Verneinung von A: $\neg A$ (auch $\bar{A})$, \textit{nicht A}, ist die Aussage, die genau dann wahr ist, wenn A falsch ist.

Wahrheitstafel: \qquad
\begin{tabular}{|c|c|}
\hline
A & $\neg A$ \\
\hline
1 & 0 \\
0 & 1 \\
\hline
\end{tabular}

Beispiele: 
\begin{itemize}
\item $A$: 6 ist durch 3 teilbar. (1)
\item $\neg A $: 6 ist nicht durch 3 teilbar. (0)
\item $B$: 4,5 ist eine gerade Zahl (0)
\item $\neg B$: 4,5 ist keine gerade Zahl. (1)
\end{itemize}

\subsection{Konjunktion}
Verknüpfung von A und B durch \textit{und}: $A \wedge B$ ist genau dann wahr, wenn A und B gleichzeitig wahr sind.

Wahrheitstafel: \qquad
\begin{tabular}{|c c |c|}
\hline
A & B & $A \wedge B$ \\
\hline
1 & 1 & 1 \\
1 & 0 & 0 \\
0 & 1 & 0 \\
0 & 0 & 0 \\
\hline
\end{tabular}

Beispiele:
\begin{itemize}
\item $\underbrace{\text{6 ist eine gerade Zahl}}_{A (1)}$ und $\underbrace{\text{durch 3 teilbar}}_{B (1)}$. (1)
\item $\underbrace{\text{9 ist eine gerade Zahl}}_{A (0)}$ und $\underbrace{\text{durch 3 teilbar}}_{B (1)}$. (0)
\end{itemize}

\subsection{Disjunktion}
\textit{oder}: $A \lor B$

Wahrheitstafel: \qquad
\begin{tabular}{|c c |c|}
\hline
A & B & $A \lor B$ \\
\hline
1 & 1 & 1 \\
1 & 0 & 1 \\
0 & 1 & 1 \\
0 & 0 & 0 \\
\hline
\end{tabular}

{\fontencoding{U}\fontfamily{futs}\selectfont\char 66\relax} Einschließendes oder, kein entweder...oder.

Beispiele:
\begin{itemize}
\item 6 ist gerade oder durch 3 teilbar. (1)
\item 9 ist gerade oder durch 3 teilbar. (1)
\item 7 ist gerade oder durch 3 teilbar. (0)
\end{itemize}

\subsection{ XOR}
\textit{entweder oder}: A xor B, $A \oplus B$ (ausschließendes oder, exclusive or).

Wahrheitstafel: \qquad
\begin{tabular}{|c c |c|}
\hline
A & B & $A \oplus B$ \\
\hline
1 & 1 & 0 \\
1 & 0 & 1 \\
0 & 1 & 1 \\
0 & 0 & 0 \\
\hline
\end{tabular}

\subsection{Implikation}
\textit{wenn, dann}, $A \Rightarrow B$:
\begin{itemize}
\item wenn A gilt, dann auch B
\item A impliziert B
\item aus A folgt B
\item A ist \underline{hinreichend} für B,
\item B ist \underline{notwendig} für A
\end{itemize}

Wahrheitstafel: \qquad
\begin{tabular}{|c c |c|}
\hline
A & B & $A \Rightarrow B$ \\
\hline
1 & 1 & 1 \\
1 & 0 & 0 \\
0 & 1 & 1 \\
0 & 0 & 1 \\
\hline
\end{tabular}

% TODO
% "ex falso quodlibet" : aus einer falschen Aussage kann man alles folgern!

(Die Implikation $A \Rightarrow B$ sagt nur, dass B wahr sein muss, \underline{falls} A wahr ist. Sie sagt nicht, dass B tatsächlich war ist.)

Beispiele:
\begin{itemize}
\item Wenn 1 = 0, bin ich der Papst. (1)
\end{itemize}

\subsection{Äquivalenz}
\textit{genau dann wenn}, $ A \Leftrightarrow B$ (dann und nur dann wenn, g.d.w, äquivalent, if and only if, iff)

Wahrheitstafel: \qquad
\begin{tabular}{|c c |c|}
\hline
A & B & $A \Leftrightarrow B$ \\
\hline
1 & 1 & 1 \\
1 & 0 & 0 \\
0 & 1 & 0 \\
0 & 0 & 1 \\
\hline
\end{tabular}

Beispiele:
\begin{itemize}
\item Heute ist Montag genau dann wenn morgen Dienstag ist. (1)
\item Eine natürliche Zahl $\underbrace{\text{ist durch 6 teilbar}}_{A}$ g. d. w. sie $\underbrace{\text{durch 3 teilbar ist}}_{B}$. (0) 
$A \Rightarrow B$ (1) 

$B \Rightarrow A$ (0)
\end{itemize}




\end{document}