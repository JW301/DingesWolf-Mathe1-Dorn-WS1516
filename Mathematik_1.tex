% !TEX TS-program = pdflatex
% !TEX encoding = UTF-8 Unicode

\documentclass[a4paper, 12pt, twoside] {article}

\usepackage[utf8]{inputenc} % set input encoding (not needed with XeLaTeX)
\usepackage[ngerman]{babel}

\usepackage{amssymb} % math stuff
\usepackage{amsmath} % math stuff
\usepackage{multicol}

\usepackage{geometry} 
\usepackage{graphicx}

% Overwrite symbols in author footnotes
\makeatletter
\let\@fnsymbol\@arabic
\makeatother

\usepackage[parfill]{parskip} % Activate to begin paragraphs with an empty line rather than an indent

%%% PACKAGES
\usepackage{booktabs} % for much better looking tables
\usepackage{array} % for better arrays (eg matrices) in maths
\usepackage{paralist} % very flexible & customisable lists (eg. enumerate/itemize, etc.)
\usepackage{verbatim} % adds environment for commenting out blocks of text & for better verbatim
\usepackage{subfig} % make it possible to include more than one captioned figure/table in a single float
% These packages are all incorporated in the memoir class to one degree or another...

%%% HEADERS & FOOTERS
\usepackage{fancyhdr} % This should be set AFTER setting up the page geometry
\pagestyle{fancy} % options: empty , plain , fancy
\renewcommand{\headrulewidth}{0pt} % customise the layout...
\lhead{}\chead{}\rhead{}
\lfoot{}\cfoot{\thepage}\rfoot{}

%%% SECTION TITLE APPEARANCE
\usepackage{sectsty}
\allsectionsfont{\sffamily\mdseries\upshape} % (See the fntguide.pdf for font help)
% (This matches ConTeXt defaults)

%%% ToC (table of contents) APPEARANCE
\usepackage[nottoc,notlof,notlot]{tocbibind} % Put the bibliography in the ToC
\usepackage[titles,subfigure]{tocloft} % Alter the style of the Table of Contents
\renewcommand{\cftsecfont}{\rmfamily\mdseries\upshape}
\renewcommand{\cftsecpagefont}{\rmfamily\mdseries\upshape} % No bold!

\usepackage{wasysym}

\usepackage{tikz}

\usepackage{venndiagram}

\usepackage{mathtools}

\usepackage{commath}

\usepackage{hyperref} % This should be the last package loaded.

\hypersetup{linktoc=all,  
hidelinks}

\newcommand{\attention}{{\fontencoding{U}\fontfamily{futs}\selectfont\char 66\relax}\space}

%%% END Article customizations

%%% CONTENT starts here

\title{Mathematik I WS 15/16}
\author{Thomas Dinges\thanks{thomas.dinges@student.uni-tuebingen.de} \and Jonas Wolf \thanks{mail@jonaswolf.de}}

\begin{document}
\maketitle

\vfill
\thanks{Inoffizielles Skript für die Vorlesung Mathematik I im WS 15/16, bei Britta Dorn. Alle Angaben ohne Gewähr. Fehler können gerne via E-Mail gemeldet werden.}

\newpage

\tableofcontents

\newpage

\section{Logik}

%% 12 Oktober 2015
%% =============

\subsection*{Aussagenlogik}
Eine \textbf{logische Aussage} ist ein Satz, der entweder wahr oder falsch (also nie beides zugleich) ist. 
Wahre Aussagen haben den Wahrheitswert 1 (auch wahr, w, true, t), falsche den Wert 0 (auch falsch, f, false).

Notation: Aussagenvariablen $A, B, C, ... A_1, A_2$.

Beispiele:
\begin{itemize}
\item 2 ist eine gerade Zahl (1)
\item Heute ist Montag (1)
\item 2 ist eine Primzahl (1)
\item 12 ist eine Primzahl (0)
\item Es gibt unendlich viele Primzahlen (1)
\item Es gibt unendlich viele Primzahlzwillinge (Aussage, aber unbekannt, ob 1 oder 0)
\item 7 (keine Aussage)
\item Ist 173 eine Primzahl? (keine Aussage)
\end{itemize}

%% 14 Oktober 2015
%% =============

Aus einfachen Aussagen kann man durch logische Verknüpfungen (\textbf{Junktoren}, z.B. und, oder, ...) kompliziertere bilden. Diese werden Ausdrücke genannt (auch Aussagen sind Ausdrücke). 
Durch sogenannte \textbf{Wahrheitstafeln} gibt man an, wie der Wahrheitswert der zusammengesetzten Aussage durch die Werte der Teilaussagen bedingt ist. Im folgenden seien $A, B$ Aussagen. 

Die wichtigsten Junktoren:

\subsection{Negation}
Verneinung von A: $\neg A$ (auch $\bar{A})$, \textit{nicht A}, ist die Aussage, die genau dann wahr ist, wenn A falsch ist.

Wahrheitstafel: \qquad
\begin{tabular}{| c | c |}
\hline
A & $\neg A$ \\
\hline
1 & 0 \\
0 & 1 \\
\hline
\end{tabular}

Beispiele: 
\begin{itemize}
\item $A$: 6 ist durch 3 teilbar. (1)
\item $\neg A $: 6 ist nicht durch 3 teilbar. (0)
\item $B$: 4,5 ist eine gerade Zahl (0)
\item $\neg B$: 4,5 ist keine gerade Zahl. (1)
\end{itemize}

\subsection{Konjunktion}
Verknüpfung von A und B durch \textit{und}: $A \wedge B$ ist genau dann wahr, wenn A und B gleichzeitig wahr sind.

Wahrheitstafel: \qquad
\begin{tabular}{| c c | c |}
\hline
A & B & $A \wedge B$ \\
\hline
1 & 1 & 1 \\
1 & 0 & 0 \\
0 & 1 & 0 \\
0 & 0 & 0 \\
\hline
\end{tabular}

Beispiele:
\begin{itemize}
\item $\underbrace{\text{6 ist eine gerade Zahl}}_{A (1)}$ und $\underbrace{\text{durch 3 teilbar}}_{B (1)}$. (1)
\item $\underbrace{\text{9 ist eine gerade Zahl}}_{A (0)}$ und $\underbrace{\text{durch 3 teilbar}}_{B (1)}$. (0)
\end{itemize}

\subsection{Disjunktion}
\textit{oder}: $A \lor B$

Wahrheitstafel: \qquad
\begin{tabular}{| c c | c |}
\hline
A & B & $A \lor B$ \\
\hline
1 & 1 & 1 \\
1 & 0 & 1 \\
0 & 1 & 1 \\
0 & 0 & 0 \\
\hline
\end{tabular}

\attention Einschließendes oder, kein entweder...oder.

Beispiele:
\begin{itemize}
\item 6 ist gerade oder durch 3 teilbar. (1)
\item 9 ist gerade oder durch 3 teilbar. (1)
\item 7 ist gerade oder durch 3 teilbar. (0)
\end{itemize}

\subsection{XOR}
\textit{entweder oder}: A xor B, $A \oplus B$ (ausschließendes oder, exclusive or).

Wahrheitstafel: \qquad
\begin{tabular}{| c c | c |}
\hline
A & B & $A \oplus B$ \\
\hline
1 & 1 & 0 \\
1 & 0 & 1 \\
0 & 1 & 1 \\
0 & 0 & 0 \\
\hline
\end{tabular}

\subsection{Implikation}
\textit{wenn, dann}, $A \Rightarrow B$:
\begin{itemize}
\item wenn A gilt, dann auch B
\item A impliziert B
\item aus A folgt B
\item A ist \underline{hinreichend} für B,
\item B ist \underline{notwendig} für A
\end{itemize}

Wahrheitstafel: \qquad
\begin{tabular}{| c c | c |}
\hline
A & B & $A \Rightarrow B$ \\
\hline
1 & 1 & 1 \\
1 & 0 & 0 \\
0 & 1 & 1 \\
0 & 0 & 1 \\
\hline
\end{tabular}

% TODO
% "ex falso quodlibet" : aus einer falschen Aussage kann man alles folgern!

(Die Implikation $A \Rightarrow B$ sagt nur, dass B wahr sein muss, \underline{falls} A wahr ist. Sie sagt nicht, dass B tatsächlich war ist.)

Beispiele:
\begin{itemize}
\item Wenn 1 = 0, bin ich der Papst. (1)
\end{itemize}

\subsection{Äquivalenz}
\textit{genau dann wenn}, $ A \Leftrightarrow B$ (dann und nur dann wenn, g.d.w, äquivalent, if and only if, iff)

Wahrheitstafel: \qquad
\begin{tabular}{| c c | c |}
\hline
A & B & $A \Leftrightarrow B$ \\
\hline
1 & 1 & 1 \\
1 & 0 & 0 \\
0 & 1 & 0 \\
0 & 0 & 1 \\
\hline
\end{tabular}

Beispiele:
\begin{itemize}
\item Heute ist Montag genau dann wenn morgen Dienstag ist. (1)
\item Eine natürliche Zahl $\underbrace{\text{ist durch 6 teilbar}}_{A}$ g. d. w. sie $\underbrace{\text{durch 3 teilbar ist}}_{B}$. (0) 
$A \Rightarrow B$ (1) 

$B \Rightarrow A$ (0)
\end{itemize}

%% 19 Oktober 2015
%% =============

\subsection*{Festlegung}
$\neg$ bindet stärker als alle anderen Junktoren: $(\neg A \wedge B)$ heißt $ (\neg A) \wedge B$

\subsection{Beispiel}
\subsubsection*{a)}
Wann ist der Ausdruck $(A \lor B) \wedge \neg (A \wedge B)$ wahr?

$\rightarrow$ Wahrheitstafel

\begin{tabular}{| c c | c | c | c | c |}
\hline
A & B & $(A \lor B)$ & $(A \wedge B)$ & $\neg (A \wedge B)$  & $(A \lor B) \wedge \neg (A \wedge B)$ \\
\hline
1 & 1 & 1 & 1 & 0 & 0 \\
1 & 0 & 1 & 0 & 1 & 1 \\
0 & 1 & 1 & 0 & 1 & 1 \\
0 & 0 & 0 & 0 & 1 & 0 \\
\hline
\end{tabular}

\attention Klammerung relevant

Welche Wahrheitswerte ergeben sich für
\begin{itemize}
\item $A \lor (B \wedge \neg A) \wedge B)$?
\item $A \lor B \wedge \neg A \wedge B$?
\end{itemize}

$(A \lor B) \wedge \neg (A \wedge B)$ und $(A \oplus B)$ haben dieselben Wahrheitstafeln.
Ausdrücke sehen unterschiedlich aus (Syntax), aber haben dieselbe Bedeutung (Semantik). Dies führt zu \textit{1.8 Definition}.

\subsubsection*{b)}
Wann ist $(A \wedge B) \Rightarrow \neg (C \lor A)$ falsch?

$\rightarrow$ Wahrheitstafel:
\underline{alle} möglichen Belegungen von $A, B, C$ mit $0 / 1$

\begin{tabular}{| c c c | c | c | c |}
\hline
A & B & C & $(A \wedge B)$ & $\neg(C \lor A)$ & $(A \wedge B) \Rightarrow \neg (C \lor A)$ \\
\hline
1 & 1 & 1 & 1 & 0 & 0 \\
1 & 1 & 0 & 1 & 0 & 0 \\
1 & 0 & 1 & 0 & 0 & 1 \\
1 & 0 & 0 & 0 & 0 & 1 \\
0 & 1 & 1 & 0 & 0 & 1 \\
0 & 1 & 0 & 0 & 1 & 1 \\
0 & 0 & 1 & 0 & 0 & 1 \\
0 & 0 & 0 & 0 & 1 & 1 \\
\hline
\end{tabular}

oder überlegen:

$(A \wedge B) \Rightarrow \neg  (C \lor A)$ ist nur 0, wenn

\qquad $(A \wedge B) = 1$, also $A = 1$ und $B = 1$

und

\qquad $\neg(C \lor A) = 0$ ist.

(Wissen: $A = 1$), also $\underline{C = 0}$ oder $ \underline{C = 1}$ möglich. 

\subsection{Definition}

Haben zwei Ausdrücke $\alpha$ und $\beta$ bei jeder Kombination von Wahrheitswerten ihrer Aussagevariablen den gleichen Wahrheitswert, so heißen sie \underline{logisch äquivalent}; man schreibt $\alpha \equiv \beta$. ('$\equiv$' ist kein Junktor, entspricht '$=$')

Es gilt: Falls $\alpha \equiv \beta$ gilt, hat der Ausdruck $\alpha \Leftrightarrow \beta$ immer den Wahrheitswert $1$.

\subsection{Satz}

Seien $A$, $B$, $C$ Aussagen.
Es gelten folgende logische Äquivalenzen:
\begin{description}
  \item[a) Doppelte Negation:]
  $A \equiv \neg(\neg A)$

  \item[b) Kommutativität von $\wedge$, $\lor$, $\oplus$, $\Leftrightarrow$:] \hfill
  \begin{itemize}
    \item $(A \wedge B) \equiv (B \wedge A)$
    \item $(A \lor B) \equiv (B \lor A)$
    \item $(A \oplus B) \equiv (B \oplus A)$
    \item $(A \Leftrightarrow B) \equiv (B \Leftrightarrow A)$

    \attention gilt nicht für '$\Rightarrow$' !! ($A \Rightarrow B \not\equiv B \Rightarrow A$)
  \end{itemize}  

  \item[c) Assoziativität von $\wedge$, $\lor$, $\oplus$, $\Leftrightarrow$:] \hfill
  \begin{itemize}
      \item $(A \wedge B) \wedge C \equiv A \wedge (B \wedge C)$
      \item $(A \lor B) \lor C \equiv A \lor (B \lor C)$
      \item $(A \oplus B) \oplus C \equiv A \oplus (B \oplus C)$
      \item $(A \Leftrightarrow B) \Leftrightarrow C \equiv A \Leftrightarrow (B \Leftrightarrow C)$
  \end{itemize}

  \item[d) Distributivität:] \hfill
  \begin{itemize}
  \item $A \wedge (B \lor C) \equiv (A \wedge B) \lor (A \wedge C)$
  \item $A \lor (B \wedge C) \equiv (A \lor B) \wedge (A \lor C)$
  \end{itemize}

  \item[e) Regeln von DeMorgan:] \hfill
  \begin{itemize}
  \item $\neg (A \wedge B) \equiv \neg A \lor \neg B$
  \item $\neg (A \lor B) \equiv \neg A \wedge \neg B$
  \end{itemize}

  \item[f)]
  $A \Rightarrow B \equiv \neg B \Rightarrow \neg A$

  \item[g)]
  $A \Rightarrow B \equiv \neg A \lor B$

  \item[h)]
  $A \Leftrightarrow B \equiv (A \Rightarrow B) \wedge (B \Rightarrow A)$
\end{description}
  (Alle Äquivalenzen gelten auch, wenn die Aussagevariablen durch Ausdrücke ersetzt werden.)

\underline{Beweis:} Jeweils mittels Wahrheitstafel (Übung!), zum Beispiel:

a) \qquad
\begin{tabular}{| c | c | c |}
\hline
A & $\neg A$ & $\neg (\neg A)$ \\
\hline
1 & 0 & 1 \\
0 & 1 & 0 \\
\hline
\end{tabular}

%% MISSING: arrows to show identity of columns 0 and 2

e) \qquad
\begin{tabular}{| c c | c | c | c | c | c |}
\hline
A & B & $(A \wedge B)$ & $\neg (A \wedge B)$ & $\neg A$ & $\neg B$ & $(\neg A \lor \neg B)$ \\
\hline
1 & 1 & 1 & 0 & 0 & 0 & 0 \\
1 & 0 & 0 & 1 & 0 & 1 & 1 \\
0 & 1 & 0 & 1 & 1 & 0 & 1 \\
0 & 0 & 0 & 1 & 1 & 1 & 1 \\
\hline
\end{tabular}

%% MISSING: arrows to show identity of columns 3 and 6

\subsection{Bemerkung}
(1.9 f): $(A \Rightarrow B) \equiv \underbrace{(\neg B \Rightarrow \neg A)}_{\mathrlap{\text{wird \underline{Kontraposition} genannt, wichtig für Beweis. Wird im Sprachgebrauch oft falsch verwendet.}}}$

\hfill

\textbf{Beispiel:} $\underset{A}{\text{Pit ist ein Dackel.}} \Rightarrow \underset{B}{\text{Pit ist ein Hund.}}$

äquivalent zu: $(\neg B) \Rightarrow (\neg A)$

\qquad Pit ist kein Hund. $\Rightarrow$ Pit ist kein Dackel.

aber nicht zu: $B \Rightarrow A$

\qquad Pit ist ein Hund. $\Rightarrow$ Pit ist ein Dackel.

und nicht zu: $\neg A \Rightarrow \neg B$

\qquad Pit ist kein Dackel. $\Rightarrow$ Pit ist kein Hund.

\textbf{Beispiel:} Sohn des Logikers / bellende Hunde ($\rightarrow$ Folien)

\subsection{Bemerkung (Logisches Umformen)}
Sei $\alpha$ ein Ausdruck. Ersetzen von Teilausdrücken von $\alpha$ durch logisch äquivalente Ausdrücke liefert einen zu $\alpha$ äquivalenten Ausdruck. So erhält man eventuell kürzere/einfachere Ausdrücke, zum Beispiel:

$\neg (A \Rightarrow B) \underset{\text{1.9 g})}{\equiv} \neg (\neg A \lor B) \underset{\text{1.9 e)}}{\equiv} \neg (\neg A) \wedge (\neg B) \underset{\text{1.9 a)}}{\equiv} A \wedge \neg B$

%% 21 Oktober 2015
%% =============

\subsection{Definition}
Ein Ausdruck heißt \underline{Tautologie}, wenn er für jede Belegung seiner Aussagevariablen, immer den Wert 1 annimmt. Hat er immer Wert 0, heißt er \underline{Kontradiktion}. 
Gibt es mindestens eine Belegung der Aussagevariablen, so dass der Ausdruck Wert 1 hat, heißt er \underline{erfüllbar}.

\subsection{Beispiel}
\begin{itemize}
\item[a)] $A \lor \neg A$ Tautologie \newline $A \wedge \neg A$ Kontradiktion

\item[b)] $\neg (A \Rightarrow B ) \Leftrightarrow A \wedge \neg B$ Tautologie (vergleiche Beispiel in 1.11). \newline
$(A \Rightarrow B) \Leftrightarrow (\neg A \lor B)$ Tautologie (vergleiche Beispiel in 1.9g).

\item[c)] $A \wedge \neg B$ ist erfüllbar (durch $A = 1, B = 0$).
\end{itemize}

\subsection*{Prädikatenlogik}
Eine \underline{Aussageform} ist ein sprachliches Gebilde, dass formal wie eine Aussage aussieht, aber eine oder mehrere Variablen enthält.

Beispiel:
$P(x): \underbrace{x}_{Variable} \underbrace{< 10}_{\mathrlap{\text{Prädikat (Eigenschaft)}}}$

$Q(x): x$ studiert Informatik
$R(y): y$ ist Primzahl und $y^2+2$ ist Primzahl.

Eine Aussageform$P(x)$ wird zur Aussage, wenn man die Variable durch ein konkretes Objekt ersetzt. Diest ist nur dann sinnvoll, wenn klar ist, welche Werte für x erlaubt sind, daher wird oft die zugelassene Wertemenge mit angegeben. (hier Vorgriff auf Kapitel \textit{Mengen})

Im Beispiel:

$P(3)$ ist wahr, $P(42)$ falsch.

$R(2)$ ist falsch, $R(3)$ ist wahr.

Oft ist die Frage interessant, ob es wenigstens ein $x$ gibt, für das $P(x)$ wahr ist, oder ob $P(x)$ sogar für alle zugelassenen $x$ wahr ist.

\subsection{Definition}
Sei $P(x)$ eine Aussageform.

a) Die Aussage \textit{Für alle x (aus einer bestimmten Menge M) gilt $P(x)$.} ist wahr genau dann wenn $P(x)$ für alle in Frage kommenden $x$ wahr ist.

Schreibweise: $\underbrace{\forall}_{\text{für alle, für jedes}} x \underbrace{\in M}_{\text{aus der Menge M}} \underbrace{:}_{\text{gilt}} \underbrace{P(x)}_{\text{Eigenschaft}}$

auch $\underbrace{\forall}_{x \in M} P(x)$.

Das Symbol $\forall$ heißt All- Quantor, die Aussage All- Aussage.

b) Die Aussage \textit{Es gibt (mindestens) ein x aus M, das die Eigenschaft P(x) besitzt.} ist wahr, g.d.w P(x) für mindestens eines der in Frage kommenden x wahr ist.

Schreibweise: $\underbrace{\exists}_{\text{es gibt, es existiert}} x \in M \underbrace{:}_{\text{so dass gilt}} P(x)$.

$\exists$ heißt Existenzquantor, die Aussage Existenzmenge.

\subsection{Beispiel / Bemerkung}
Übungsgruppe G:
$\underbrace{a}_{Anna} \underbrace{b}_{Bob} \underbrace{c}_{Clara}$

$B(x): x$ ist blond.
$W(x): x$ ist weiblich.

$B(a) = 1, W(b) = 0)$

\begin{enumerate}

\item Alle Studenten der Gruppe sind blond. (1)

$\forall x \in G$: x ist blond

$\forall x \in G$: B(x) (1)

Das bedeutet:
a blond $\wedge$ b blond $\wedge$ c blond \newline
$\underbrace{B(a)}_{1} \wedge \underbrace{B(b)}_{1} \wedge \underbrace{B(c)}_{1}$

$\forall$ ist also eine Verallgemeinerung der Konjunktion.

\item Alle Studenten der Gruppe sind weiblich. (0)

$\underbrace{W(a)}_{1} \wedge \underbrace{W(b)}_{0} \wedge \underbrace{W(c)}_{1} (0)$

\item Es gibt einen Studenten der Gruppe, der weiblich ist. (1)

$\exists x \in G$: W(x) (1)

bedeutet: $\underbrace{W(a)}_{1} \lor \underbrace{W(b)}_{0} \lor \underbrace{W(c)}_{1} = 1$

$\exists$ ist verallgemeinerte Disjunktion.

\item Aussage A: Alle Studenten der Gruppe sind weiblich. (0)

Verneinung von A? $\neg A$

\attention Nicht korrekt wäre: Alle Studenten der Gruppe sind männlich. (Wahrheitswert ist auch 0)

Korrekt: Nicht alle Studenten der Gruppe sind weiblich (1)
Es gibt (mindestens) einen Studenten der Gruppe, der nicht weiblich ist. (1)

\end{enumerate}

allgemeiner:

\subsection{Negation von All- und Existenzaussagen}

\begin{itemize}
\item[a)] $\neg (\forall x \in M: P(x)) \equiv \exists x \in M: \neg P(x)$
\item[b)] $\neg (\exists x \in M: P(x)) \equiv \forall x \in M : \neg P(x)$
\end{itemize}

(Verallgemeinerung der Regeln von DeMorgan)
(vergleiche Beispiel 1.15, 4):

$\neg (\forall x \in G: W(x))$

$\equiv \neg (W(a) \wedge W(b) \wedge W(c)$

$\underbrace{\equiv}_{\mathrlap{DeMorgan}} (\neg W(a)) \lor (\neg W(b)) \lor (\neg (W(c))$

$\equiv \exists x \in G: \neg W(x)$

%% 26 Oktober 2015
%% =============

\subsection*{Bemerkung}
Aussageformen können auch mehrere Variablen enthalten, Aussagen mit mehreren Quantoren sind möglich.

Zum Beispiel:

$\exists x \in X \quad \exists y \in Y: P(x,y)$ \\
$\exists x \in X \quad \forall y \in Y: P(x,y)$ \\
$\forall x \in X \quad \exists y \in Y: P(x,y)$ \\
$\forall x \in X \quad \forall y \in Y: P(x,y)$

Negation dann durch mehrfaches Anwenden von 1.16, zum Beispiel:

$\neg (\forall x \in X \quad \forall y \in Y \quad \exists z \in Z : P(x,y,z))$ \\
$\equiv \exists x \in X : \neg (\forall y \in Y \quad \exists z \in Z : P(x,y,z))$ \\
$\equiv \exists x \in X \quad \exists y \in Y : \neg (\exists z \in Z : P(x,y,z))$ \\
$\equiv \exists x \in X \quad \exists y \in Y \quad \forall z \in Z : \neg P(x,y,z))$

\textbf{Also: }\\
ändere $\exists$ in $\forall$, \\
\text{\qquad \quad} $\forall$ in $\exists$, \\
verneine Prädikat.


\section{Mengen}

\subsection{Definition (Georg Cantor, 1845-1918)}

Eine \underline{Menge} ist eine Zusammenfassung von bestimmten wohlunterscheidbaren Objekten (\underline{Elementen}) unserer Anschauung oder unseres Denkens zu einem Ganzen.

Im Folgenden seien $A$, $B$ Mengen.

\begin{description}
\item[a)] 
	$\quad x \in A : x \text{ ist Element der Menge } A$ \\
	$x \notin A: x \text{ ist nicht Element der Menge } A$ \\
	oder auch: \\
	$A \ni x : x \text{ ist Element der Menge } A$ \\
	$A \not \ni x: x \text{ ist nicht Element der Menge } A$
\item[b)]
	Eine Menge kann beschrieben werden durch:
	\begin{itemize}
		\item Aufzählung ihrer Elemente, zum Beispiel: \\
		$M_1 = \{a,b,c\} \qquad \text{(}=\{c,a,b\} \text{, d.h. Reihenfolge spielt keine Rolle)}$ \\
		\textbf{Achtung:} Keine Wiederholungen! \\
		$M_2 = \{\smiley,\frownie\}$ \\
		$M_3 = \{ \underline{3}, \underline{\{1,2\}}, \underline{M_1}\}$ \\
		geht nur bei endlichen Mengen oder bestimmten unendlichen Mengen, zum Beispiel: \\
		$\mathbb{N} = \{1,2,3,4,...\}$ Menge der natürlichen Zahlen \\
		$\mathbb{N}_0 = \{1,2,3,4,...\}$ Menge der natürlichen Zahlen mit der Null \\
		$\mathbb{Z} = \{0,1,-1,2,-2,...\}$ Menge der ganzen Zahlen
		\item Charakterisierung ihrer Elemente: \\
		$A = \{x \mid x \text{ besitzt die Eigenschaft } E\}$, z.B.:\\
		$A = \{n \underbrace{\mid}_{\mathrlap{\text{sprich: \textit{''mit der Eigenschaft''}}}} n \in \mathbb{N} \text{ und n ist gerade}\}$\\
		$\quad = \{2,4,6,8,...\}$ \\
		$\quad = \{ x \mid \exists k \in \mathbb{N} \text{ mit } x = 2 \cdot k\} = \{2k \mid k \in \mathbb{N}\}$ \\
		
		Bsp: $\mathbb{Q} = \{\frac{a}{b} \mid a,b \in \mathbb{Z}, b \neq 0 \}$ Menge der rationalen Zahlen		
	\end{itemize}
\item[c)]
	Mit $\emptyset$ bezeichnen wir die Menge ohne Elemente (\underline{leere Menge})
\item[d)]
	Mit $\abs{A}$ bezeichnen wir die Anzahl der Elemente der Menge $A$ (\underline{Kardinalität} oder \underline{Mächtigkeit} von $A$), zum Beispiel: \\
	$\abs{\{1,a,*\}} = 3, \quad \abs{\emptyset} = 0, \quad \abs{\mathbb{N}} = \infty, \quad \abs{\{\mathbb{N}\}} = 1$
\item[e)]
	$A \cap B \underbrace{:=}_{\mathrlap{\text{wird definiert als}}} \{x \mid x \in A \wedge x \in B\}$ heißt \underline{Durchschnitt} oder \underline{Schnittmenge} von $A$ und $B$.
	
	Grafische Veranschaulichung: Venn-Diagramm (\attention gilt nicht als Beweis)
	
	\begin{venndiagram2sets}
	\fillACapB
	\end{venndiagram2sets}

	
\item[f)]
	$A \cup B :=\{x \mid x \in A \lor x \in B \}$ heißt \underline{Vereinigung} von $A$ und $B$.
		
	\begin{venndiagram2sets}
	\fillA \fillB
	\end{venndiagram2sets}
	
\item[Beispiele:]
	$A = \{1,2,3\}$, $B = \{2,3,4\}$, $C = \{4\}$\\ \\
		$A \cap B = \{2,3\}$,\\
		$A \cap C = \emptyset$,\\
		$B \cap C = \{4\} = C$,\\
		$A \cup B = \{1,2,3,4\}$
		
\item[g)]
	$A$ und $B$ heißen \underline{disjunkt}, falls gilt $A \cap B = \emptyset$
		
	\begin{venndiagram2sets}[overlap=-20]		
	\end{venndiagram2sets}
	
\item[h)]
	$A$ heißt \underline{Teilmenge} von $B$, $A \subseteq B$, falls gilt: \\
	$x \in A \Rightarrow x \in B$\\
	Oder in Worten: Jedes Element von $A$ ist auch Element von $B$.
	
	Dasselbe bedeutet die Notation\\
	$B \supseteq A$ \\
	($B$ ist Obermenge von $A$)
	
	Beispiel: $\{1,2\} \subseteq \{1,2,3\} \subseteq \mathbb{N} \subseteq \mathbb{N}_0 \subseteq \mathbb{Z} \subseteq \mathbb{R}$ (reelle Zahlen)
	
	Es gilt: $\emptyset \subseteq A$ für jede Menge $A$.
\end{description}

\textbf{Achtung: } Unterschied $\subseteq, \in$ !\\
Zum Beispiel: \\
$A = \{1, \mathbb{N}\}$ (hier ist die Menge $\mathbb{N}$ ein Element von A, keine Teilmenge!)\\
$1 \in A, \qquad \mathbb{N} \in A, \qquad \mathbb{N} \nsubseteq A, \qquad 2 \notin A, \qquad \{1\} \subseteq A$

%% 28 Oktober 2015
%% =============


\end{document}